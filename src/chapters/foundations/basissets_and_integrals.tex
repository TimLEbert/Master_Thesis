In order to Represent orbitals inside a Kohn Sham DFT calculation atom centric basis sets are the norm.
These basis sets usually consist of a radial part times a radial distribution function.
\subsection{Polarized Atomic Orbitals}
Polarized Atomic Orbitals(PAOs) are derived as a linear combination from a larger primary basis of functions centered on individuals atom centers. Formally we can write a PAO basis function $\eta_{\mu}(\mathbf{r})$ as a weighted sum of the primary basis functions $\tilde\eta_\nu(\mathbf{r})$ where $\mu,\nu$ are belonging to the same Atom.
\begin{align}
    \tilde \eta_{\mu} &=\Sum\limits_{\nu} A_{\mu,\nu} \eta_{\nu} 
\end{align}
\subsection{Gaussian basis functions}
The the radial individal basis functions $\eta_{\mu}(\mathbf{r})$ are usually chosen either as Gaussian
's centered on the same atom at position($\mathbf{R}$) with different exponents and multiplied with spherical harmonics$Y_{lm}$ to accomplish some angular dependency.
\begin{align}
    \eta_{\mu}(\mathbf{r}) &= N(\alpha,m,l) ||\mathbf{r}-\mathbf{R}||^l Y_{lm}(\frac{\mathbf{r}-\mathbf{R}}{||\mathbf{r}-\mathbf{R}||}) e^{-\alpha ||\mathbf{r}-\mathbf{R}||^2}
\end{align}
The advantages of Gaussian orbitals are that they are computationally efficient, are easily integrated and the product of two gaussians is again a gaussian which simplifies many otherwise very complex integrals.
The spherical harmonics on the other side enables arbitrary angular resolution and can be formulated in their Cartesian Formulation as an polynominal of just order $l$.
\begin{align}
    ||\mathbf{v}||^l Y_{lm}(\frac{\mathbf{v}}{||\mathbf{v}||}) &= \Sum\limits_{a,b,c\in \mathbb{N}_0 a+b+c\leq l} A_{l,a,b,c} v_1^a v_2^b v_3^c
\end{align}
Where $A_{l,a,b,c}$ are constants. This leads to PAOs just being finite sums of finite polynominals multiplied with gaussians.

\subsection{Integrals}
There are several integrals in Kohn Sham dft which are nessesary to compute. All of them can be formulated in the following way.
\begin{align}
    I_{\mu,\nu} = \int d\mathbf{r} \eta_{\mu}(\mathbf{r}) f(\mathbf{r}) \eta_{\nu}(\mathbf{r})
\end{align}
Where $f$ is a function of $\mathbf{r}$. The integral can be simplified by the fact that the product of two gaussians is again a gaussian. This leads to the integral being a finite sum of gaussians multiplied with polynominals of the type of the spherical harmonics.

