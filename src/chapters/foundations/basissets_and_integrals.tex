In order to Represent orbitals inside a Kohn Sham DFT calculation atom centric basis sets are the norm.
These basis sets usually consist of a radial part times a radial distribution function.
\subsection{Polarized Atomic Orbitals}
Polarized Atomic Orbitals(PAOs) are derived as a linear combination from a larger primary basis of functions centered on individuals atom centers. Formally we can write a PAO basis function $\eta_{\mu}(\mathbf{r})$ as a weighted sum of the primary basis functions $\tilde\eta_\nu(\mathbf{r})$ where $\mu,\nu$ are belonging to the same Atom.
\begin{align}
    \tilde \eta_{\mu} &=\sum\limits_{\nu} A_{\mu,\nu} \eta_{\nu} 
\end{align}
\subsection{Gaussian basis functions}
The the radial individal basis functions $\eta_{\mu}(\mathbf{r})$ are usually chosen either as Gaussian
's centered on the same atom at position($\mathbf{R}$) with different exponents and multiplied with spherical harmonics$Y_{lm}$ to accomplish some angular dependency.
\begin{align}
    \eta_{\mu}(\mathbf{r}) &= N(\alpha,m,l) ||\mathbf{r}-\mathbf{R}||^l Y_{lm}(\frac{\mathbf{r}-\mathbf{R}}{||\mathbf{r}-\mathbf{R}||}) e^{-\alpha ||\mathbf{r}-\mathbf{R}||^2}
\end{align}
The advantages of Gaussian orbitals are that they are computationally efficient, are easily integrated and the product of two gaussians is again a gaussian which simplifies many otherwise very complex integrals.
The spherical harmonics on the other side enables arbitrary angular resolution and can be formulated in their Cartesian Formulation as an polynominal of just order $l$.
\begin{align}
    ||\mathbf{v}||^l Y_{lm}(\frac{\mathbf{v}}{||\mathbf{v}||}) &= \sum\limits_{a,b,c\in \mathbb{N}_0 a+b+c\leq l} A_{l,a,b,c} v_1^a v_2^b v_3^c
\end{align}
Where $A_{l,a,b,c}$ are constants. This leads to PAOs just being finite sums of finite polynominals multiplied with gaussians.

\subsection{Integrals}
Most relevant integrals for DFT involve integrals over products of gaussians which can be computed analytically. Examples are the overlap integral $\langle \eta_{\mu}|\eta_{\nu}\rangle$, the kinetic energy integral $\langle \nabla \eta_{\mu}|\nabla \eta_{\nu}\rangle$, the electron-nucleus integral $\langle \eta_{\mu}|\frac{Z}{||\mathbf{r}-\mathbf{R}||}|\eta_{\nu}\rangle$ as well as the hartree integral $(\eta_{\mu}\eta_{\nu}|\eta_\gamma\eta_\delta)$. Other more complicated integrals, such as the Thomas Fermi functional or the von Weizsäcker functional, as well as practically all exchange correlation functionals, are usually computed numerically on a sufficently large grid.
The way these integrals are usually calculated can be demonstrated on the example of the overlap integral. First one can notice that it is possible to write the product of the two gaussians as a single gaussian and simplify.
\begin{align}
    \langle \phi|\psi \rangle &= \langle   \sum\limits_{a,b,c\in \mathbb{N}_0 a+b+c\leq l_1} A_{l,a,b,c} (v_1-x_1)^a (v_2-x_2)^b (v_3-x_3)^c e^{-\alpha (\mathbf v -\mathbf x)^2}| \sum\limits_{a,b,c\in \mathbb{N}_0 d+e+f\leq l_2} B_{l,d,e,f} (w_1-x_1)^d (w_2-x_2)^e (w_3-x_3)^{f}e^{-\beta(\mathbf w -\mathbf x)^2}\rangle\\
    &=\int \sum\limits_{a,b,c\in \mathbb{N}_0 a+b+c\leq l_1+l_2} C_{l,a,b,c} (x_1)^a (x_2)^b (x_3)^{c}e^{-(\alpha+\beta)\alpha(\frac{\alpha \mathbf v +\beta\mathbf w}{\alpha+\beta} -\mathbf x)^2-\frac{\alpha\beta}{\alpha+\beta}(\mathbf{v}-\mathbf{w})^2}\\
    &=\sum\limits_{a,b,c\in \mathbb{N}_0 a+b+c\leq l_1+l_2} C_{l,a,b,c} I_{a}(\alpha,\beta,(\mathbf{v}-\mathbf{w})|_{1})I_{b}(\alpha,\beta,(\mathbf{v}-\mathbf{w})|_{2})I_{c}(\alpha,\beta,(\mathbf{v}-\mathbf{w})|_{3})
\end{align}
The individual integrals $I_{a}(\alpha,\beta,x)$ can be computed using recursion relations the recursion root can be camputed as:
\begin{align}
    I_{0}(\alpha,\beta,\mathbf{v}-\mathbf{w}) &= \sqrt{\frac{\pi}{\alpha+\beta}} e^{-\frac{\alpha\beta}{\alpha+\beta}(\mathbf{v}-\mathbf{w})^2}
\end{align}
And the higher order terms as:
\begin{align}
    I_{a,b,c}(\alpha,\beta,\mathbf{v}-\mathbf{w}) &= \frac{1}{2(\alpha+\beta)}\left( (a-1)I_{a-1,b,c}(\alpha,\beta,\mathbf{v}-\mathbf{w}) + (b-1)I_{a,b-1,c}(\alpha,\beta,\mathbf{v}-\mathbf{w})\right.\\
    &\left.+ (c-1)I_{a,b,c-1}(\alpha,\beta,\mathbf{v}-\mathbf{w})\right)
\end{align}
There exist more recursion relations for the other types of integrals.