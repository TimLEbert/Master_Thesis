In order to represent spacial functions $f:\mathbb{R}^3\right \mathbb{C}$ inside a DFT calculation atom centric basis sets are most commonly used, as they provide anhanced resolution around the nuleii where most of the electron density is concentrated.
\subsection{Contracted atomic orbitals}
Contracted atomic orbitals are defined as a linear combination of a larger primary basis of basis functions centered on atom centers. Formally we can write a contracted atomic orbitals basis function $\eta_{\mu}(\mathbf{r})$ as a weighted sum of the primary basis functions $\tilde\eta_\nu(\mathbf{r})$ where $\mu,\nu$ are belonging to the same Atom.
\begin{align}
    \eta_{\mu} &=\sum\limits_{\nu} A_{\mu,\nu} \tilde \eta_{\nu}
\end{align}
\subsection{Contracted Gaussian Type Orbitals (CGTOs)}
The radial distribution of primary basis functions $\eta_{\mu}(\mathbf{r})$ is typically chosen as Gaussians centered on the same atom at position $\mathbf{R}$ with different exponents, multiplied by spherical harmonics $Y_{lm}$ to introduce angular dependency.
\begin{align}
\eta_{\mu}(\mathbf{r}) &= N(\alpha,m,l) ||\mathbf{r}-\mathbf{R}||^l Y_{lm}(\frac{\mathbf{r}-\mathbf{R}}{||\mathbf{r}-\mathbf{R}||}) e^{-\alpha ||\mathbf{r}-\mathbf{R}||^2}
\end{align}
where $N(\alpha,m,l)$ are normalisation constants.
Gaussian orbitals offer several advantages: they are computationally efficient, easily integrated, and the product of two Gaussians remains a Gaussian, which simplifies many otherwise complex integrals. However, a single Gaussian cannot accurately describe the electron density cusp near the nucleus. To address this, primitive Gaussians are typically contracted to Contracted Gaussian-Type Orbitals (CGTOs) to form a more physically representative basis set. The contraction coefficients $A_{\mu,\nu}$ are fitted to more accurate basis functions like Slater-type orbitals (STOs).
Spherical harmonics provide arbitrary angular resolution and can be formulated in their Cartesian representation as a polynomial of order $l$.
\begin{align}
    ||\mathbf{v}||^l Y_{lm}(\frac{\mathbf{v}}{||\mathbf{v}||}) &= \sum\limits_{a,b,c\in \mathbb{N}_0 a+b+c\leq l} A_{l,a,b,c} v_1^a v_2^b v_3^c
\end{align}
Where $A_{l,a,b,c}$ are constants. This leads to CGTOs just being finite sums of finite polynominals multiplied with gaussians.

\subsection{Integrals}
Most relevant integrals for DFT involve integrals over products of gaussians which can be computed analytically. Examples are the overlap integral $\langle \eta_{\mu}|\eta_{\nu}\rangle$, the kinetic energy integral $\langle \nabla \eta_{\mu}|\nabla \eta_{\nu}\rangle$, the electron-nucleus integral $\langle \eta_{\mu}|\frac{Z}{||\mathbf{r}-\mathbf{R}||}|\eta_{\nu}\rangle$ as well as the hartree integral $(\eta_{\mu}\eta_{\nu}|\eta_\gamma\eta_\delta)$. While these closed form solutions exist they still involve many repeated recursion relations and can be very compute intensive for larger molecules. Other more complicated integrals, such as the Thomas Fermi functional or the von Weizsäcker functional, as well as practically all exchange correlation functionals, are usually computed numerically on a sufficently large grid. In this work we are using the implementationa of the libcint library \cite{sun_libcint_2015} to compute these integrals, when they don't have to be differentiable.

