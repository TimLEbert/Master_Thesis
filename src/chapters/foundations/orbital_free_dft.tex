Orbital-Free Density Functional Theory (OF-DFT) represents a significant branch in the evolution of electronic structure methods. Rooted in the original formulation of Density Functional Theory by Hohenberg and Kohn \cite{HohenbergKohn1964}, OF-DFT aims to calculate the ground state properties of a many-electron system directly from the electron density, without the need for individual electronic orbitals.
\subsection{Theoretical Foundation}

Based on equation ref{eq:1} the total energy of a system can be expressed as a functional of the electron density $n(\mathbf{r})$. The chalenge of Orbtial free DFT lies in finding an accurate representation of the kinetic energy functional $T_s[n]$ directly from the density, without resorting to orbitals. Historic attemps at this culminated in the Thomas-Fermi functional and the von Weizsäcker functional. These functionals were based on the kinetic energy of a non-interacting electron gas and its gradient.

\subsubsection{Thomas-Fermi Functional}

The simplest approximation is the Thomas-Fermi functional:

\begin{equation}
T_{TF}[n] = C_{TF} \int n^{5/3}(\mathbf{r}) d\mathbf{r}
\end{equation}

where $C_{TF} = \frac{3}{10}(3\pi^2)^{2/3}$.

\subsubsection{von Weizsäcker Functional}

The von Weizsäcker functional adds a gradient correction:

\begin{equation}
T_{vW}[n] = \frac{1}{8} \int \frac{|\nabla n(\mathbf{r})|^2}{n(\mathbf{r})} d\mathbf{r}
\end{equation}

\subsubsection{Generalized Gradient Approximation (GGA)}

More sophisticated functionals incorporate higher-order density gradients:

\begin{equation}
T_{GGA}[n] = \int t_{GGA}(n(\mathbf{r}), \nabla n(\mathbf{r}), \nabla^2 n(\mathbf{r}), ...) d\mathbf{r}
\end{equation}

If One is able to represent the total Energy of the Systen as a function of the density one is able to find the gound state by minimizing the total energy functional using conventional techniges like gradient decent or SLSQC.
But with the rise of Machine Learning in the field of DFT, a promising new approach to this problem is to learn an emperical approximation of the kinetic energy functional from a large dataset of electronic structure calculations. \cite{remmmmme}\cite{zhang_m-ofdft_2023}
But both these Attemps failed to extrapolate to larger systems and were not able to model the functional around the ground state fine enough to let the minimization procedere converge on the true ground state.
To train an machine learning model, first labels for the kinetic energy functional are needed. As \cite{zhang_m-ofdft_2023} proposed learning the energies and the gradient of the model, which is used for the minimization of the total energy functional, should lead to a more realistic model.
MOFDFT\cite{zhang_m-ofdft_2023} also proposed to used the minimization procedure in the kohn sham step to generate labels at every scf iteration. Furthermore Manuel proposed to peturb the fock matrix of the scf iterations with gasussian noise to generate more diverse labels for the model which lay further away from the ground state.

he label for the kinetic energy in these coordinates is indirectly calculated by:

\begin{align}
    T_S(\mathbf{p}) = T_S(C) + E_{H}(C) + E_{XC}(C) + E_{ext}(C) - E_{H}(\mathbf{p}) + E_{XC}(\mathbf{p}) + E_{ext}(\mathbf{p}).
\end{align}

We cannot calculate the gradient of the kinetic energy directly from this equation as the density :math:`p` depends on
the coefficients in the orbital basis $C$ in a nontrivial way.
Instead, we make use of the minimization procedure in the ksdft calculation. Each iteration the energy is minimized as follows:

\begin{align}
    \phi^{k} = \underset{\{\phi_i\}_{i=1}^n \text{orthonormal}}{\text{argmin}} \langle \psi_{\mathbf{\phi}} | \hat T_S | \psi_{\mathbf{\phi}} \rangle + \sum_{k'<k} \pi^{(k')}_k V_{eff}^{k'}[\rho_{[\mathbf{\phi}]}] + V_{peturb}^{k}(\rhoPhi)
\end{align}


Where $\pi^{(i)}_k$ are the DIIS coefficients of the KSDFT calculation and

\begin{align}
    V_{eff}^{k'}[\rho_{[\mathbf{\phi}]}] = \int \rho_{[\mathbf{\phi}]}(\mathbf{r}) V_{eff[\rho_{[\mathbf{\phi}^{k'}]}]}(r)dr\\
    V_{peturb}^{k}(\rhoPhi) = \int \rhoPhi(\mathbf{r}) v_{peturb,\mu}^{k}\omega_\mu(\mathbf{r})dr\\
\end{align}
is the effective potential of the $k'$-th iteration integrated over the density of the new density and the pertubation of the effective potential of the $k$th iteration. With $v_{peturb,\mu}^{k} \sim \epsilon_k * \mathcal{N}(1,0)$ and $\omega_\mu$ being basis function in the orbital free basis.
$\epsilon_k$ is the chosen such that starts at the 5 interation and decreases over time.
This is solved using Laplace multipliers:

\begin{align}
    \frac{\delta T_S[\rho_{[\mathbf{\phi}^{k}]}]}{\delta \rho}(\mathbf{r}) + \sum_{k'<k} \pi^{(k')}_k V_{eff}^{k'}(\mathbf{r}) + v_{peturb,\mu}^{k}\omega_\mu(\mathbf{r})= \mu^{(k)}
\end{align}
The projected gradient of the kinetic energy is then calculated as the DIIS weighted average over the projected last few effective potentials:

\begin{align}
    \nabla_\mathbf{p} T_S(\mathbf{p}) &= \int \frac{\delta T_S[\rho_{[\mathbf{\phi}^{k}]}]}{\delta \rho}(\mathbf{r}) \mathbf{\omega}(\mathbf{r}) d\mathbf{r} = -\int\sum_{k'<k} \pi^{(k')}_k V_{eff}^{k'}(\mathbf{r}) \mathbf{\omega}v_{peturb,\mu}^{k}\omega_\mu(\mathbf{r})(\mathbf{r}) +\mu^{(k)}\mathbf{\omega}(\mathbf{r})d\mathbf{r} = -{\mathbf{v}}_{eff\{\mathbf{p}^{k'}\}_{k'< k}}+\mu^{(k)}\mathbf{w}\\
    \left( \textbf{I}-\frac{\textbf{w}^{(d)}{\textbf{w}^{(d)}}^T}{{\textbf{w}^{(d)}}^T
        \textbf{w}^{(d)}}\right) \nabla_\mathbf{p} T_S(\mathbf{p}) &= -\left( \textbf{I}-\frac{\textbf{w}^{(d)}{\textbf{w}^{(d)}}^T}{{\textbf{w}^{(d)}}^T
        \textbf{w}^{(d)}}\right){\mathbf{v}}_{eff\{\mathbf{p}^{k'}\}_{k'< k}}\\
    \mathbf{v}_{eff}(\mathbf{p}) &= \nabla_\mathbf{p} (E_{H}(\mathbf{p}) + E_{XC}(\mathbf{p}) + E_{ext}(\mathbf{p}))
\end{align}
The exchange-correlation energy $E_{xc}[n]$ is typically approximated using functionals similar to those in Kohn-Sham DFT, such as LDA or GGA.
The calculation of the effective potential in terms of the orbital free density we first need to fit the kohn sham density to the orbitals.
\subsection{Density fitting}
This procedere called density fitting is also very commonly done to speed up kohn sham calculation , as the four center coulomb integral used to calculate the hartree energy scales with $O(N^4)$ and is therefore very expensive for large systems. As for this usecase it is only important that the hartree energy of the produced by the fitted basis is accurate the Hartree energy of the residual density is often the metric of choocise to minimise.
\begin{align}
        \mathcal{L}(\mathbf{p}) &= \mathbf{p} \tilde{W} \mathbf{p} - 2 \mathbf{p}\bar {\tilde L} \bar\Gamma + \bar\Gamma \tilde{\mathbf{D}}\bar\Gamma
\end{align}
By variational methods this equation can be minimized in the following way:
\begin{align}
\partial_{\mathbf p}\mathcal L&= 2\tilde{W} \mathbf{p}- 2 \bar {\tilde L} \bar\Gamma=0\\
\mathbf{p}&=\tilde{W}^{-1}\bar {\tilde L} \bar\Gamma\\
&=\bar{\mathbf{P}} \bar\Gamma
\end{align}
Which results in a equation that can be easily solved by least squares methods.
But this simple method results in bad in inaccurate densities and bad other energies.
In Section ... better density fitting methods are discussed and compared.
\subsection{Preparation of the data}
\subsubsection{Natrep}
\subsubsection{Dimensionwise rescaling}
\subsubsection{Local frames}
