\subsection{Kohn Sham Density Functional Theory(KS-DFT)}
The exact formulation of the Hamiltonial of the electrons of a static many body system, like a molecule, taken into account the born oppenheimer approximation of static nucleii, is given by the Schrödinger equation(using natural units):


\begin{align}
H[\mathbf{\Psi}] = T(\mathbf{\Psi}) + V_{ee}(\mathbf{\Psi}) + V_{ext}(\mathbf{\Psi})
\end{align}
Where
\begin{align}
    T[\mathbf{\Psi}] &= \frac{1}{2}\int\nabla\mathbf{\Psi}^*(\mathbf{r})\nabla\mathbf{\Psi}(\mathbf{r}) dr \\
    V_{ee}[\mathbf{\Psi}] &= \frac{1}{2}\int\int\frac{1}{|\mathbf{r}-\mathbf{r'}|}\mathbf{\Psi}^*(\mathbf{r})\mathbf{\Psi}^*(\mathbf{r}')\mathbf{\Psi}(\mathbf{r}')\mathbf{\Psi}(\mathbf{r})d\mathbf{r}d\mathbf{r'}\\
    V_{ext}[\mathbf{\Psi}] &= \int v_{ext}(\mathbf{r})\mathbf{\Psi}^*(\mathbf{r})\mathbf{\Psi}(\mathbf{r})d\mathbf{r}
\end{align}

Here $\mathbf{Psi}$ is


 \cite{KohnSham1965}, KSDFT extends the foundational work of Hohenberg and Kohn \cite{HohenbergKohn1964} on Density Functional Theory (DFT).
\begin{align}
    E^{\star}=\min _{\psi: \text { antisym },\langle\psi \mid \psi\rangle=1}\langle\psi| \hat{T}+\hat{V}_{\mathrm{ee}}+\hat{V}_{\mathrm{ext}}|\psi\rangle\\
    \rho_{[\psi]}(\mathbf{r}):=N \int\left|\psi\left(\mathbf{r}, \mathbf{r}^{(2)}, \cdots, \mathbf{r}^{(N)}\right)\right|^2 \mathrm{~d} \mathbf{r}^{(2)} \cdots \mathrm{d} \mathbf{r}^{(N)}\\
\end{align}
$\begin{aligned} E^{\star} & =\min _{\rho: \geqslant 0, \int \rho(\mathbf{r}) \mathrm{d} \mathbf{r}=N}\left(\min _{\psi: \text { antisym }, \rho_{[\psi]}=\rho}\langle\psi| \hat{T}+\hat{V}_{\mathrm{ee}}|\psi\rangle\right)+E_{\text {ext }}[\rho] \\ & =\min _{\rho: \geqslant 0, \int \rho(\mathbf{r}) \mathrm{d} \mathbf{r}=N}\left\{E[\rho]:=U[\rho]+E_{\text {ext }}[\rho]\right\} .\end{aligned}$
...
This leads to the celebrated Kohn Sham equations:

\begin{align}
    H_{KS}[\rho] = T_S[\rho] + V_{eff}[\rho]
\end{align}
Where

\begin{align}
    T_S[\rho] &:=\min _{\phi_i: \text {orthorgonal, normalized}, \rho_{[\phi]}=\rho }\sum\limits_{i=1}^n\nabla\phi_i \nabla \phi_i\\
    V_{eff}[\rho] &:= V_{ext}[\rho] + V_H[\rho] + V_{xc}[\rho]\\
    V_H[\rho] &:= \int\int \frac{\rho(\mathbf{r})\rho(\mathbf{r'})}{|\mathbf{r}-\mathbf{r'}|}d\mathbf{r}d\mathbf{r'}\\
    V_{xc}[\rho] &:= \int \rho(\mathbf{r})\epsilon_{xc}(\rho(\mathbf{r}))d\mathbf{r}\\
    V_{ext}[\rho] &:= \int v_{ext}(\mathbf{r})\rho(\mathbf{r})d\mathbf{r}.
\end{align}
The Kohn Sham equations are solved iteratively using the Self-Consistent Field (SCF) method:
The total energy is expressed in terms of the Kohn-Sham orbitals as:
\begin{align}
    E_{KS} = \min _{\phi_i: \text {orthorgonal, normalized}} \left( \sum_{i=1}^N \phi_i \Delta \phi_i + E_\text{H}[\rhoPhi]+ E_\text{XC}[\rhoPhi]+E_\text{ext}[\rhoPhi]\right)
\end{align}
To solve the kohn sham equations the Energy is varied with respect to the orbitals and the resulting equations are solved iteratively using the SCF method.
\begin{align}
    \frac{\delta E_{KS}}{\delta \phi_i(\mathbf{r})} = -\Delta \phi_i(\mathbf{r}) + \frac{\delta E_\text{H}[\rhoPhi]}{\delta \phi_i}(\mathbf{r}) + \frac{\delta E_\text{XC}[\rhoPhi]}{\delta \phi_i}(\mathbf{r}) + \frac{\delta E_\text{ext}[\rhoPhi]}{\delta \phi_i}(\mathbf{r}) = 0
\end{align}
As the exchange correlation energy and the hartree energy are more complicated functionals of the density, this equation can't be solved analytically and is instead solved iteratively using the self consistend field method.
\subsubsection{Self Consistent Field (SCF)Method}
In the SCF method the following steps are repeated until convergence is reached:
We will from now on begin to enumerate the orbitals corresponding to their interation in the scf procedure $\mathbf{\Phi}^{(j)} = \{\phi_i^{(j)}\}_{i=1}^n$.
\begin{enumerate}
    \item Start with an initial guess for $\rho(\mathbf{r})$
    \item Calculate $V_{eff}(\mathbf{r})$
    \item Solve the Kohn-Sham equations to obtain $\phi_i(\mathbf{r})$
    \item Calculate a new density: $\rho_{new}(\mathbf{r}) = \sum_i |\phi_i(\mathbf{r})|^2$
    \item If $|\rho_{new}(\mathbf{r}) - \rho(\mathbf{r})| < \text{tolerance}$, stop. Otherwise, return to step 2 with $\rho(\mathbf{r}) = \rho_{new}(\mathbf{r})$
    \item Calculate the total energy $E_{KS}$
    \item If $|E_{KS} - E_{KS_{old}}| < \text{tolerance}$, stop. Otherwise, return to step 2 with $\rho(\mathbf{r}) = \rho_{new}(\mathbf{r})$
    \item Calculate the forces on the nuclei
    \item Update the positions of the nuclei
    \item Return to step 1
    \item If the forces on the nuclei are below a certain threshold, stop.
\end{enumerate}
If we want now to apply this precedere to a real system we also have to take into account the way that we want to represent the orbitals. In the following we will introduce the concept of basis sets and how they are used to represent the orbitals in a kohn sham calculation.



















