\subsection{Historical Background}
Kohn-Sham Density Functional Theory (KSDFT) stands as a cornerstone in computational quantum mechanics, revolutionizing our approach to electronic structure calculations. Developed by Walter Kohn and Lu Jeu Sham in 1965 \cite{KohnSham1965}, KSDFT extends the foundational work of Hohenberg and Kohn \cite{HohenbergKohn1964} on Density Functional Theory (DFT).
\subsection{Theoretical Foundation}
The theoretical underpinning of DFT rests on two fundamental theorems proved by Hohenberg and Kohn:
\begin{theorem}
The ground-state properties of a many-electron system are uniquely determined by the electron density $n(\mathbf{r})$.
\end{theorem}
\begin{theorem}
There exists a universal functional of the electron density, $F[n(\mathbf{r})]$, which can be used to find the ground-state energy of the system.
\end{theorem}
These theorems establish that the ground-state energy of a system can be expressed as a functional of the electron density:
\begin{equation}
E[n] = F[n] + \int V_{\text{ext}}(\mathbf{r})n(\mathbf{r})d\mathbf{r}
\end{equation}
where $V_{\text{ext}}(\mathbf{r})$ is the external potential and $F[n]$ is a universal functional independent of the external potential.
\subsection{The Kohn-Sham Approach}
Kohn and Sham proposed a practical approach to apply DFT by introducing a fictitious system of non-interacting particles that generate the same density as the system of interacting particles. This approach involves solving a set of single-particle Schrödinger-like equations, known as the Kohn-Sham equations:
\begin{equation}
\left[-\frac{1}{2}\nabla^2 + V_{\text{eff}}(\mathbf{r})\right]\phi_i(\mathbf{r}) = \epsilon_i\phi_i(\mathbf{r})
\end{equation}
where $\phi_i(\mathbf{r})$ are the Kohn-Sham orbitals and $\epsilon_i$ are their corresponding energies. The effective potential $V_{\text{eff}}(\mathbf{r})$ is defined as:
\begin{equation}
V_{\text{eff}}(\mathbf{r}) = V_{\text{ext}}(\mathbf{r}) + V_{\text{H}}(\mathbf{r}) + V_{\text{xc}}(\mathbf{r})
\end{equation}
Here, $V_{\text{ext}}(\mathbf{r})$ is the external potential, $V_{\text{H}}(\mathbf{r})$ is the Hartree potential, and $V_{\text{xc}}(\mathbf{r})$ is the exchange-correlation potential.
\subsection{Contributions to the Kohn-Sham Energy}
The total energy in the Kohn-Sham formulation can be expressed as:
\begin{equation}
E_{\text{KS}} = T_s[n] + E_{\text{H}}[n] + E_{\text{xc}}[n] + E_{\text{ext}}[n]
\end{equation}
Let us examine each term in detail:
\subsubsection{Kinetic Energy of Non-interacting Electrons}
The kinetic energy of the non-interacting electrons, $T_s[n]$, is given by:
\begin{equation}
T_s[n] = -\frac{1}{2}\sum_i \int \phi_i^*(\mathbf{r})\nabla^2\phi_i(\mathbf{r})d\mathbf{r}
\end{equation}
This term represents the kinetic energy of the Kohn-Sham orbitals.
\subsubsection{Hartree Energy}
The Hartree energy, $E_{\text{H}}[n]$, represents the classical electrostatic interaction energy of the electron density:
\begin{equation}
E_{\text{H}}[n] = \frac{1}{2}\int\int \frac{n(\mathbf{r})n(\mathbf{r'})}{|\mathbf{r}-\mathbf{r'}|}d\mathbf{r}d\mathbf{r'}
\end{equation}
\subsubsection{Exchange-Correlation Energy}
The exchange-correlation energy, $E_{\text{xc}}[n]$, encapsulates all many-body effects beyond the Hartree approximation. Its exact form is unknown, and developing accurate approximations for this term is a central challenge in DFT. Common approximations include:
\begin{itemize}
\item Local Density Approximation (LDA):
\begin{equation}
E_{\text{xc}}^{\text{LDA}}[n] = \int n(\mathbf{r})\epsilon_{\text{xc}}(n(\mathbf{r}))d\mathbf{r}
\end{equation}
where $\epsilon_{\text{xc}}(n)$ is the exchange-correlation energy per particle of a uniform electron gas of density $n$.
\item Generalized Gradient Approximation (GGA):
\begin{equation}
E_{\text{xc}}^{\text{GGA}}[n] = \int f(n(\mathbf{r}), |\nabla n(\mathbf{r})|)d\mathbf{r}
\end{equation}
where $f$ is a function of both the density and its gradient.
\end{itemize}
\subsection{External Potential Energy}
The external potential energy, $E_{\text{ext}}[n]$, represents the interaction of the electrons with the external potential (typically due to the nuclei):
\begin{equation}
E_{\text{ext}}[n] = \int V_{\text{ext}}(\mathbf{r})n(\mathbf{r})d\mathbf{r}
\end{equation}
\section{Self-Consistent Field Method}
The Kohn-Sham equations are solved iteratively using the Self-Consistent Field (SCF) method:
\begin{enumerate}
\item Start with an initial guess for $n(\mathbf{r})$
\item Calculate $V_{\text{eff}}(\mathbf{r})$
\item Solve the Kohn-Sham equations to obtain $\phi_i(\mathbf{r})$
\item Calculate a new density: $n_{\text{new}}(\mathbf{r}) = \sum_i |\phi_i(\mathbf{r})|^2$
\item If $|n_{\text{new}}(\mathbf{r}) - n(\mathbf{r})| < \text{tolerance}$, stop. Otherwise, return to step 2 with $n(\mathbf{r}) = n_{\text{new}}(\mathbf{r})$
\end{enumerate}
