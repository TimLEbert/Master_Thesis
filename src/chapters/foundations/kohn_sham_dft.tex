\subsection{Kohn Sham Density Functional Theory(KS-DFT)}
The exact formulation of the Hamiltonial of the electrons of a static many body system, like a molecule, taken into account the born oppenheimer approximation of static nucleii, is given by the Schrödinger equation(using natural units):


\begin{align}
H[\mathbf{\Psi}] = T(\mathbf{\Psi}) + V_{ee}(\mathbf{\Psi}) + V_{ext}(\mathbf{\Psi})
\end{align}
Where
\begin{align}
    T[\mathbf{\Psi}] &= \frac{1}{2}\int\nabla\mathbf{\Psi}^*(\mathbf{r})\nabla\mathbf{\Psi}(\mathbf{r}) dr \\
    V_{ee}[\mathbf{\Psi}] &= \frac{1}{2}\int\int\frac{1}{|\mathbf{r}-\mathbf{r'}|}\mathbf{\Psi}^*(\mathbf{r})\mathbf{\Psi}^*(\mathbf{r}')\mathbf{\Psi}(\mathbf{r}')\mathbf{\Psi}(\mathbf{r})d\mathbf{r}d\mathbf{r'}\\
    V_{ext}[\mathbf{\Psi}] &= \int v_{ext}(\mathbf{r})\mathbf{\Psi}^*(\mathbf{r})\mathbf{\Psi}(\mathbf{r})d\mathbf{r}
\end{align}

Here $\mathbf{Psi}$ is


 \cite{KohnSham1965}, KSDFT extends the foundational work of Hohenberg and Kohn \cite{HohenbergKohn1964} on Density Functional Theory (DFT).
\begin{align}
    E^{\star}=\min _{\psi: \text { antisym },\langle\psi \mid \psi\rangle=1}\langle\psi| \hat{T}+\hat{V}_{\mathrm{ee}}+\hat{V}_{\mathrm{ext}}|\psi\rangle\\
    \rho_{[\psi]}(\mathbf{r}):=N \int\left|\psi\left(\mathbf{r}, \mathbf{r}^{(2)}, \cdots, \mathbf{r}^{(N)}\right)\right|^2 \mathrm{~d} \mathbf{r}^{(2)} \cdots \mathrm{d} \mathbf{r}^{(N)}\\
\end{align}
$\begin{aligned} E^{\star} & =\min _{\rho: \geqslant 0, \int \rho(\mathbf{r}) \mathrm{d} \mathbf{r}=N}\left(\min _{\psi: \text { antisym }, \rho_{[\psi]}=\rho}\langle\psi| \hat{T}+\hat{V}_{\mathrm{ee}}|\psi\rangle\right)+E_{\text {ext }}[\rho] \\ & =\min _{\rho: \geqslant 0, \int \rho(\mathbf{r}) \mathrm{d} \mathbf{r}=N}\left\{E[\rho]:=U[\rho]+E_{\text {ext }}[\rho]\right\} .\end{aligned}$
...
This leads to the celebrated Kohn Sham equations:

\begin{align}
    H_{KS}[\rho] = T_S[\rho] + V_{eff}[\rho]
\end{align}
Where

\begin{align}
    T_S[\rho] &:=\min _{\phi_i: \text {orthorgonal, normalized}, \rho_{[\phi]}=\rho }\sum\limits_{i=1}^n\nabla\phi_i \nabla \phi_i\\
    V_{eff}[\rho] &:= V_{ext}[\rho] + V_H[\rho] + V_{xc}[\rho]\\
    V_H[\rho] &:= \int\int \frac{\rho(\mathbf{r})\rho(\mathbf{r'})}{|\mathbf{r}-\mathbf{r'}|}d\mathbf{r}d\mathbf{r'}\\
    V_{xc}[\rho] &:= \int \rho(\mathbf{r})\epsilon_{xc}(\rho(\mathbf{r}))d\mathbf{r}\\
    V_{ext}[\rho] &:= \int v_{ext}(\mathbf{r})\rho(\mathbf{r})d\mathbf{r}.
\end{align}
The Kohn Sham equations are solved iteratively using the Self-Consistent Field (SCF) method:
The total energy is expressed in terms of the Kohn-Sham orbitals as:
\begin{align}
    E_{KS} = \min _{\phi_i: \text {orthorgonal, normalized}} \left( \sum_{i=1}^N \phi_i \Delta \phi_i + E_\text{H}[\rhoPhi]+ E_\text{XC}[\rhoPhi]+E_\text{ext}[\rhoPhi]\right)
\end{align}
To solve the kohn sham equations the Energy is varied with respect to the orbitals and the resulting equations are solved iteratively using the SCF method.
\begin{align}
    \frac{\delta E_{KS}}{\delta \phi_i(\mathbf{r})} = -\Delta \phi_i(\mathbf{r}) + \frac{\delta E_\text{H}[\rhoPhi]}{\delta \phi_i}(\mathbf{r}) + \frac{\delta E_\text{XC}[\rhoPhi]}{\delta \phi_i}(\mathbf{r}) + \frac{\delta E_\text{ext}[\rhoPhi]}{\delta \phi_i}(\mathbf{r}) = 0
\end{align}
As the exchange correlation energy and the hartree energy are more complicated functionals of the density, this equation can't be solved analytically and is instead solved iteratively using the self consistend field method.
\subsubsection{Self Consistent Field (SCF)Method}
In the SCF method the following steps are repeated until convergence is reached:
We will from now on begin to enumerate the orbitals corresponding to their interation in the scf procedure $\mathbf{\Phi}^{(j)} = \{\phi_i^{(j)}\}_{i=1}^n$.
\begin{enumerate}
    \item Start with an initial guess for $\rho(\mathbf{r})$
    \item Calculate $V_{eff}(\mathbf{r})$
    \item Solve the Kohn-Sham equations to obtain $\phi_i(\mathbf{r})$
    \item Calculate a new density: $\rho_{new}(\mathbf{r}) = \sum_i |\phi_i(\mathbf{r})|^2$
    \item If $|\rho_{new}(\mathbf{r}) - \rho(\mathbf{r})| < \text{tolerance}$, stop. Otherwise, return to step 2 with $\rho(\mathbf{r}) = \rho_{new}(\mathbf{r})$
    \item Calculate the total energy $E_{KS}$
    \item If $|E_{KS} - E_{KS_{old}}| < \text{tolerance}$, stop. Otherwise, return to step 2 with $\rho(\mathbf{r}) = \rho_{new}(\mathbf{r})$
    \item Calculate the forces on the nuclei
    \item Update the positions of the nuclei
    \item Return to step 1
    \item If the forces on the nuclei are below a certain threshold, stop.
\end{enumerate}
If we want now to apply this precedere to a real system we also have to take into account the way that we want to represent the orbitals. In the following we will introduce the concept of basis sets and how they are used to represent the orbitals in a kohn sham calculation.



















\subsection{Theoretical Foundation}
The theoretical underpinning of DFT rests on two fundamental theorems proved by Hohenberg and Kohn:
\begin{theorem}
The ground-state properties of a many-electron system are uniquely determined by the electron density $n(\mathbf{r})$.
\end{theorem}
\begin{theorem}
There exists a universal functional of the electron density, $F[n(\mathbf{r})]$, which can be used to find the ground-state energy of the system.
\end{theorem}
These theorems establish that the ground-state energy of a system can be expressed as a functional of the electron density:
\begin{equation}
E[n] = F[n] + \int V_{\text{ext}}(\mathbf{r})n(\mathbf{r})d\mathbf{r}
\end{equation}
where $V_{\text{ext}}(\mathbf{r})$ is the external potential and $F[n]$ is a universal functional independent of the external potential.
\subsection{The Kohn-Sham Approach}
Kohn and Sham proposed a practical approach to apply DFT by introducing a fictitious system of non-interacting particles that generate the same density as the system of interacting particles. This approach involves solving a set of single-particle Schrödinger-like equations, known as the Kohn-Sham equations:
\begin{equation}
\left[-\frac{1}{2}\nabla^2 + V_{\text{eff}}(\mathbf{r})\right]\phi_i(\mathbf{r}) = \epsilon_i\phi_i(\mathbf{r})
\end{equation}
where $\phi_i(\mathbf{r})$ are the Kohn-Sham orbitals and $\epsilon_i$ are their corresponding energies. The effective potential $V_{\text{eff}}(\mathbf{r})$ is defined as:
\begin{equation}
V_{\text{eff}}(\mathbf{r}) = V_{\text{ext}}(\mathbf{r}) + V_{\text{H}}(\mathbf{r}) + V_{\text{xc}}(\mathbf{r})
\end{equation}
Here, $V_{\text{ext}}(\mathbf{r})$ is the external potential, $V_{\text{H}}(\mathbf{r})$ is the Hartree potential, and $V_{\text{xc}}(\mathbf{r})$ is the exchange-correlation potential.
\subsection{Contributions to the Kohn-Sham Energy}
The total energy in the Kohn-Sham formulation can be expressed as:
\begin{equation}
E_{\text{KS}} = T_s[n] + E_{\text{H}}[n] + E_{\text{xc}}[n] + E_{\text{ext}}[n]
\end{equation}
Let us examine each term in detail:
\subsubsection{Kinetic Energy of Non-interacting Electrons}
The kinetic energy of the non-interacting electrons, $T_s[n]$, is given by:
\begin{equation}
T_s[n] = -\frac{1}{2}\sum_i \int \phi_i^*(\mathbf{r})\nabla^2\phi_i(\mathbf{r})d\mathbf{r}
\end{equation}
This term represents the kinetic energy of the Kohn-Sham orbitals.
\subsubsection{Hartree Energy}
The Hartree energy, $E_{\text{H}}[n]$, represents the classical electrostatic interaction energy of the electron density:
\begin{equation}
E_{\text{H}}[n] = \frac{1}{2}\int\int \frac{n(\mathbf{r})n(\mathbf{r'})}{|\mathbf{r}-\mathbf{r'}|}d\mathbf{r}d\mathbf{r'}
\end{equation}
\subsubsection{Exchange-Correlation Energy}
The exchange-correlation energy, $E_{\text{xc}}[n]$, encapsulates all many-body effects beyond the Hartree approximation. Its exact form is unknown, and developing accurate approximations for this term is a central challenge in DFT. Common approximations include:
\begin{itemize}
\item Local Density Approximation (LDA):
\begin{equation}
E_{\text{xc}}^{\text{LDA}}[n] = \int n(\mathbf{r})\epsilon_{\text{xc}}(n(\mathbf{r}))d\mathbf{r}
\end{equation}
where $\epsilon_{\text{xc}}(n)$ is the exchange-correlation energy per particle of a uniform electron gas of density $n$.
\item Generalized Gradient Approximation (GGA):
\begin{equation}
E_{\text{xc}}^{\text{GGA}}[n] = \int f(n(\mathbf{r}), |\nabla n(\mathbf{r})|)d\mathbf{r}
\end{equation}
where $f$ is a function of both the density and its gradient.
\end{itemize}
\subsection{External Potential Energy}
The external potential energy, $E_{\text{ext}}[n]$, represents the interaction of the electrons with the external potential (typically due to the nuclei):
\begin{equation}
E_{\text{ext}}[n] = \int V_{\text{ext}}(\mathbf{r})n(\mathbf{r})d\mathbf{r}
\end{equation}
\section{Self-Consistent Field Method}
The Kohn-Sham equations are solved iteratively using the Self-Consistent Field (SCF) method:
\begin{enumerate}
\item Start with an initial guess for $n(\mathbf{r})$
\item Calculate $V_{\text{eff}}(\mathbf{r})$
\item Solve the Kohn-Sham equations to obtain $\phi_i(\mathbf{r})$
\item Calculate a new density: $n_{\text{new}}(\mathbf{r}) = \sum_i |\phi_i(\mathbf{r})|^2$
\item If $|n_{\text{new}}(\mathbf{r}) - n(\mathbf{r})| < \text{tolerance}$, stop. Otherwise, return to step 2 with $n(\mathbf{r}) = n_{\text{new}}(\mathbf{r})$
\end{enumerate}
