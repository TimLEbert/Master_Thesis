As mentioned in section \ref{into_density_fitting}, density fitting is a method that attempts to reproduce the electron density generated by a set of orbitals $\rho'(\mathbf{r}) = \rhoPhi(\mathbf{r})$ and physical properties thereof, using an orbital-free basis set $\rho(\mathbf{r}) = p^\mu \omega_\mu(\mathbf{r})$. It is commonly used in DFT calculations to reduce the computational cost of the 4-center hartree integral $\tilde{D}_{\alpha,\beta,\gamma,\delta}  = (\eta_\alpha\eta_\beta|\eta_\gamma\eta_\delta)$, by approximating it with the 2-center integral $\tilde{W}_{\mu\nu} = (\omega_\mu |\omega_\nu)$ between two fitted densities. This approach, often called the \textsc{resolution of the identity}(RI) method emerged in the late 1960s pioneer by Whitten\cite{whitten1973} and Dunlap\cite{dunlap1979} as a means to speed up Hartree - Fock and post Hartree - Fock calculations.
For these use cases the only quantity of the fitted density which was of interest was the hartree energy which is why the RI method mostly just involved a fit which minimizes the residual Hartree energy $H_H[\rho-\rho']$.\\
But in OF-DFT we will use the fitted density to form labels from which the ML- model is supposed to learn a functional of the total energy. This is why we need to consider the fitted density as a whole and not just the hartree energy. This is we curated a set of promising density fitting methods and prepared a varied set of metrics on which to compare the electronic properties of the fitted densities.\\
\section{Metrics} \label{metrics}
To evaluate the various density fitting methods, the following metrics are used:
\begin{enumerate}
    \item Mean Absolute Error (MAE) of the Energies (Hartree, External, xc, kinetic).
    \item MAE of the of the number of electrons $N = \int \rho(\mathbf{r}) d\mathbf{r}$.
    \item The L2 Norm of the difference in densities $\text{L2}[\rho-\rho'] = \sqrt{\int (\rho(\mathbf{r})-\rho'(\mathbf{r}))^2 d\mathbf{r}} $.
    \item The L1 Norm of the difference in densities $\text{L1}[\rho-\rho'] = \int |\rho(\mathbf{r})-\rho'(\mathbf{r})| d\mathbf{r}$.
    \item The Gradient of the total energy at the ground state.
    \item The L1 Norm of negative values in the fitted density $\text{L1}[\rho'] = \int -\rho'(\mathbf{r})\mathbf{1}_{\rho'(\mathbf{r})<0} d\mathbf{r}$.
    \item The Hartree energy of the residual density $H_{H}[\rho-\rho'] = \int \int \frac{(\rho(\mathbf{r})-\rho'(\mathbf{r}))(\rho(\mathbf{r'})-\rho'(\mathbf{r'}))}{|\mathbf{r}-\mathbf{r'}|}d\mathbf{r}d\mathbf{r'}$.
\end{enumerate}
Additionally we are going to observe the standard deviation of the fitted coefficients to get a proxy for the stability of the methods, as an very high standard for single coefficients could hint at underlying degenerate basis.
\section{Density Fitting Methods}
We tested several promising density fitting methods using these metrics. Below, we briefly explain each method’s strengths and the metrics they optimize. The detailed derivations are provided in appendix \ref{appendix:densityfitting}.\\

The first few methods are derived from a simple Lagrangian that is minimized without any additional constraints to ensure the conservation of the total integrated density, i.e., the number of electrons.
\subsection{Overlap Density fitting}
The simplest density fitting method just minimizes the L2 norm of the residual density
\begin{align} \label{overlap_eq}
        \mathcal{L}(\mathbf{p}) &= \mathbf{p} W \mathbf{p} - 2 \mathbf{p}\bar { L} \bar\Gamma + \bar\Gamma \mathbf{D}\bar\Gamma.
\end{align}
The advantage of this model is that it is very simple and close to the original idea of density fitting. An disadvantage is that as the energies are not directly optimizes don't turn out to be very accurate in practice.
\subsection{Hartree Density fitting}
This method is commonly used in electronic structure calculations to emulate the Hartree energy of the system and the original \textsc{resolution of the identity} method introduces by Whitten\cite{whitten1973}. It minimises the hartree energy of the residual density
\begin{align}
        \mathcal{L}(\mathbf{p}) &= \mathbf{p} \tilde{W} \mathbf{p} - 2 \mathbf{p}\bar {\tilde L} \bar\Gamma + \bar\Gamma \tilde{\mathbf{D}}\bar\Gamma.
\end{align}
Similar to the overlap density fitting method, the Hartree density fitting method is simple and has a single goal in providing an accurate Hartree energy.\\
It excels in this respect but lacks accuracy in other areas.
\subsection{Hartree+External Density Fitting} \label{hartreeexternal}
A simple improvement to the previous method also includes the external energy in the loss function.
\begin{align}\label{true_df}
        \mathcal{L}(\mathbf{p}) &= \mathbf{p} \tilde{W} \mathbf{p} - 2 \mathbf{p}\bar {\tilde L} \bar\Gamma + \bar\Gamma \tilde{\mathbf{D}}\bar\Gamma + (\mathbf{p}\mathbf{v}_{ext}-\bar\Gamma \bar{V}_{ext})^2
\end{align}
But as the external energy enters the equation quadratic its accuracy tends to be reduced compared to the linear entering hartree energy.
\subsection{Hartree+External Density Fitting M-OFDFT-Version}
This density fitting method was proposed in the \textsc{M-OFDFT}-paper\cite{zhang_m-ofdft_2023} to calculate the labels for their ML scheme. It minimizes the following lagrangian:
\begin{equation}
    \mathcal{L}(\mathbf{p}) &=\left\lVert
    \begin{pmatrix}
    \tilde{W} \\
    v_{ext}^T
    \end{pmatrix}
    \mathbf{p}
    -
    \begin{pmatrix}
    \tilde{L} \bar{\Gamma} \\
    \bar{\Gamma} \bar{V}_{ext}
    \end{pmatrix}
    \right\lVert^2
\end{equation}
It is similar to the previous method but with the external energy and hartree energy entering the equation concatenated. We could observe that this leads to a better reproduced external energy in practice.
\subsection{Hartree+External Density Fitting with enforced electron number}
This method is identical to Hartree+External density fitting \ref{hartreeexternal} but the number of electrons is enforced by adding a Lagrange multiplier to the loss function.
\begin{align}
    \mathcal{L}(\mathbf{p},\mu) &= \mathbf{p} \tilde{W} \mathbf{p} - 2 \mathbf{p}\bar {\tilde L} \bar\Gamma + (\mathbf{p}\mathbf{v}_{ext}-\bar\Gamma \bar{V}_{ext})^2+\mu(\mathbf{p}\mathbf{w}-\bar\Gamma\bar S)
\end{align}
However this also lead to a decrease in the accuracy of other metrics
\subsection{Hartree Density Fitting with enforced electron number and enforced external energy} \label{both_fixed}
To further utilize lagrange multiplier , we can also enforce the external energy to be exactly reproduced.
\begin{align}
    \mathcal{L}(\mathbf{p},\mu,\nu) &= \mathbf{p} \tilde{W} \mathbf{p} - 2 \mathbf{p}\bar {\tilde L} \bar\Gamma + \nu(\mathbf{p}\mathbf{v}_{ext}-\bar\Gamma \bar{V}_{ext})+\mu(\mathbf{p}\mathbf{w}-\bar\Gamma\bar S)
\end{align}
This restriction also comes with a decrease in the accuracy of other metrics and it is debatable if it is desirable to enforce a single metric while the reproduced density is of, as this could decrease the stability of the method and lead to unphysical densities.
\subsection{Hartree+External Density Fitting M-OFDFT-Version with soft enforced electron number}
While not mentioned in the M-OFDFT paper\cite{zhang_m-ofdft_2023}, the  publicly released interference code also includes a variant of the hartree+external M-OFDFT density fitting method with a soft enforced electron number. The modified lagrangian
\begin{equation}
    \mathcal{L}(\mathbf{p}) &=\left\lVert
    \begin{pmatrix}
    \tilde{W} \\
    v_{ext}^T\\
        w^T
    \end{pmatrix}
    \mathbf{p}
    -
    \begin{pmatrix}
    \tilde{L} \bar{\Gamma} \\
    \bar{\Gamma} \bar{V}_{ext}\\
    \bar\Gamma\bar S
    \end{pmatrix}
    \right\lVert^2
\end{equation}
Includes an additional concatenated term to put emphasis on reproducing the correct integrated density.
However, in practice this reduced the stability of the method while only marginally increasing the accuracy metrics of the total number of electrons.
\subsection{Overlap Density fitting with enforced electron number and enforced external energy}
At last we also tested method \ref{both_fixed} while substituting the hartree matrix for the overlap matrix.
\begin{align}
    \mathcal{L}(\mathbf{p},\mu,\nu) &= \mathbf{p} \tilde{W} \mathbf{p} - 2 \mathbf{p}\bar {\tilde L} \bar\Gamma + \nu(\mathbf{p}\mathbf{v}_{ext}-\bar\Gamma \bar{V}_{ext})+\mu(\mathbf{p}\mathbf{w}-\bar\Gamma\bar S)
\end{align}
This leads to a metric which replicates the overall density quite well, but reglects the hartree energy.
\section{Results}
We now compare the performance of different density fitting methods using the metrics described above. For ease of reproducibility we choose the first 1,000 molecules in the QM9 dataset\cite{ramakrishnan2014quantum} to base our comparison on. Ground state densities were calculated for all molecules using Kohn-Sham with the PBE xc-functional and the "6-31G(2df,p)" basis set, then an even-tempered basis set with $\beta = 2.5$ was constructed as orbital free basis set.
\subsection{AE of energies}
The absolute error (AE) of the energies is presented in Figure
\ref{fig:AE_energies}.
\begin{figure}
    \centering
    \includegraphics[width=0.9\textwidth]{chapters/results/results_images/AE_hartree_energy_on_even_tempered_2.5_for_different_df_methods}
    \includegraphics[width=0.9\textwidth]{chapters/results/results_images/AE_xc_energy_on_even_tempered_2.5_for_different_df_methods}
    \includegraphics[width=0.9\textwidth]{chapters/results/results_images/AE_ext_energy_on_even_tempered_2.5_for_different_df_methods}
    \includegraphics[width=0.9\textwidth]{chapters/results/results_images/AE_kin_energy_on_even_tempered_2.5_for_different_df_methods}
        \caption{Boxplots(\ref{boxplots}) of the AE of Energies per electron for different density fitting methods for the first 1000 molecules in QM9.}
    \label{fig:AE_energies}
\end{figure}
We observe that the different density fitting methods generally minimize the metrics entering in their loss functions. The Hartree, Hartree + external, M-OFDFT, and its restricted versions excel at reproducing the total Hartree energy, while the vanilla Hartree + external and its variants appear limited in accuracy. The DF methods that minimize the L2 norm of the residual density perform less effectively, as expected.
\subsection{Comparison fitted denities}
For the absolute error (AE) of the exchange-correlation energy, all DF methods perform similarly since none directly optimize this metric, though the overlap methods perform slightly worse. In terms of external energy, the DF methods that directly optimize it produce reliable results. The kinetic energy is best reproduced by the Hartree + external DF method and the various M-OFDFT variants.
\begin{figure}
    \includegraphics[width=0.9\textwidth]{chapters/results/results_images/AE_density_on_even_tempered_2.5_for_different_df_methods}
    \includegraphics[width=0.9\textwidth]{chapters/results/results_images/L2_residual_densities_on_even_tempered_2.5_for_different_df_methods}
    \includegraphics[width=0.9\textwidth]{chapters/results/results_images/L1_residual_densities_on_even_tempered_2.5_for_different_df_methods}
    \includegraphics[width=0.9\textwidth]{chapters/results/results_images/L2_residual_hartree_on_even_tempered_2.5_for_different_df_methods}
    \caption{Boxplots(\ref{boxplots}) of the difference in density for different DF - methods.}
    \label{fig:Densities_df_methods}
\end{figure}
In figure \ref{fig:Densities_df_methods} we can observe the direct effect of different DF methods on the density. Lets first observe the conservation of electrons. Methods without additional constraints show an absolute error (AE) in the range of 0.01\%–0.1\%, with the overlap method performing the worst. The M-OFDFT DF method with a soft constraint improves this metric by two orders of magnitude, and all methods with a hard constraint fulfill this metric up to numerical accuracy.\\
For quality of the fit of the of- densities to the ks densities we have several metrics to observe. The L2 and L1 norm of the residual density are for all DF-methods similary well reproduced with just the soft and hart constrained M-OFDFT-versions performing slightly worse, while having a larger spread than the others. Looking the the residual Hartree energy the picture looks similar with the overlap methods performing worse while the other methods perform similarly well.\\
\begin{figure}
    \includegraphics[width=0.9\textwidth]{chapters/results/results_images/L1_negative_densities_on_even_tempered_2.5_for_different_df_methods}
    \includegraphics[width=0.9\textwidth]{chapters/results/results_images/max_abs_gradient_on_even_tempered_2.5_for_different_df_methods}
    \includegraphics[width=0.9\textwidth]{chapters/results/results_images/var_density_fitting}
    \caption{Boxplots(\ref{boxplots}) of additional metrics which can be used to judge the stability of the DF-methods.}
    \label{fig:other_df_metrics}
\end{figure}
There are a few more metric we can take a look at, depicted in figure \ref{fig:other_df_metrics} that can be worth to take a look at. \\
The L1 norm of negative values is something to look out for, as they are unphysical and are not visible in the other metrics. As we can see the negative densities are mostly quite low with only a few outliers for the constrained M-OFDFT versions and overlap methods to reach critical values.\\
The gradient of the total energy is calculated using the procedere explained in \ref{labelgen_gradient} and should be zero at the ground state. Because of this the maximum of the gradient is a meainingfull metric for the quality of density fitting. For the tested DF - methods no meaning full difference could be observed between them. The maximal ground state gradient is for all DF-methods around $2.8\cdot 10^{-3}$ mHa which is sufficiently low for density optimisation procedures to be accurate.\\
At last we are going to take a look at the standard deviation of the fitted coefficients which can be seen a metric for the stability of the method. Here the individual events are not the first 1000 molecules of QM9 but instead the individual exponents of the fitted basis set for each atom type. Where we can see that the soft constrained M-OFDFT versions has the highest variation which could hint at instabilities in the method and lead to harter to learn labels. \\
\begin{figure}
    \centering
    \includegraphics[width=1\textwidth]{chapters/results/results_images/density_fitting_slices_new}
    \caption{A visualisation of the differences in densities between the fitted and the original densities using slices of a Ethanol molecule}
    \label{fig:density_slices_df}
\end{figure}
Another way of judging the quality of a fit is to take a look at the differences in density directly. In figure \ref{fig:density_slices_df} we plotted the difference in density in the form of 2d slices which are laid through the symmetry axis of an Ethanol molecule. As most of the density of an molecule is located around the atomic nuclei, also the biggest density errors are expected to be there. \\
Here, we can also see that the differences in density between the different methods are mostly very subtle. The methods basis based on optimizing the residuals and those based on overlap seem to be most similar to each other. The lowest visible difference can be seen for the overlap methods. This is because these methods aim to provide the close fit of the densities at any point in space, which is exactly what is visualized in the plots.
\subsection{Conclusion}
In summary, we found that the distinct density fitting methods fulfill different roles. The methods based on minimizing the L2 norm of the residual density perform the best for optimizing just this metric. However, for the reproducibility of the individual energies, it is necessary to use a method based on minimizing the Hartree energy of the residual density. We found that out of the unrestricted methods, the Hartree+external M-OFDFT version performed the best all around, producing reliably good energies, as well as good and stable densities, while not producing too many negative densities. Its only caveat is that it doesn't exactly conserve the number of electrons. This drawback is fixed by the use of constrained methods, which perform a minimization while fixing the correct number of electrons, at the cost of slightly worse energies and densities. The hard constrained version of Hartree+external M-OFDFT performs very well at this job, reproducing almost the same energies as its unrestricted version with just slightly worse and more unstable densities, a higher amount of negative densities, and higher variance of the coefficients. If a low variance of coefficients and conservation of the number of electrons are of concern, then the method "Hartree fixed densities and external energy" is a good choice, as it performs very well all around with just slightly worse energies than the other methods.







