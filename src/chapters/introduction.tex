Field theories are an influential and widespread field in theoretical physics. Both quantum field theories and statistical field theories are applied with great success to describe the behavior of quantum and classical statistical many-body systems.  One of the simplest nontrivial systems treated in these fields is an $N$-dimensional vector field with quadratic self interaction, a so called O(N)-model of a $\phi^4$-theory which is used in QFT to describe systems of bosons and in statistical field theories to approximately describe Ising models. Its Lagrangian is given by
\begin{equation}\label{phi4model}
\mathcal{L}(\phi) = \frac{1}{2}\left(\partial_\mu\phi\right)^2-\frac{m}{2}\phi^2-\frac{\lambda}{4!}\left(\phi^2\right)^2
\end{equation}
Here is $m$ the \textit{bare mass} of the particle corresponding to the excitation of the field and $\lambda$ the coupling constant of the quadratic self interaction of the field.  
In this work we connect each point of the field additional to an external heat bath which is dampening the excitations of the field. The action in momentum space is given by
\begin{equation}
S[\phi] =   \frac{T}{2}\sum\limits_{n \in \mathbb{Z}}\int\limits_{\mathbb{R}^d}\frac{dq^d}{(2\pi)^d}\phi_n(\vec q)\left(\omega_n^2+\vec q^2+\gamma |\omega_n|\right)\phi_n(-\vec q)+U(\phi_m(\vec q))
\end{equation}
where we introduce the \textit{dissipation constant} $\gamma$ (derivation in appendix \ref{appendix_1}) and the potential which reads in momentum space $U(\phi_m(\vec q)) = -\frac{m}{2}\phi^2-\frac{\lambda}{4!}\left(\phi^2\right)^2$. $\omega_n$ is a Matsubara mode that which are frequently used in finite temperature field theory\cite{matsubara}. For systems of these kind the approach of the \textit{Functional Renormation Group}(FRG)\cite{Gies_2012}\cite{BOHR_2001}\cite{WETTERICH199390} has proven to be an capable tool to study  systems at thermal equilibrium and derive valuable quantities. This technique works by starting with the bare action of the system and integrating out momentum shells one at a time to arrive at a effective average action $\Gamma[\phi]$ from which all expectation values can be obtained. In practice the effective average action cannot be derived exactly but one has to do an truncation which approximates the exact average effective action. The truncation we use goes by the name of \textit{local potential approximation}(LPA)\cite{ZUMBACH1994225} which uses the effective Potential $U$ and the dissipation constant as parameters to describe the the average effective action.
$\gamma$, as a parameter of the truncation, is expected to diverge at the critical temperature $T_C$ as a result of the occurring phasetransition. Mean field theory\cite{täuber_2014} predicts that the behavior of $\gamma$ near the critical point  only dependents on the reduced temperature $\tau$ which acts as the ordering parameter of this system and its behavior follows a power law described by the critical exponent $\alpha$.
\begin{equation}
\gamma(\tau) \approx \tau^\alpha = \left(\frac{T-T_C}{T_C}\right)^\alpha
\end{equation} 
Critical exponents are important identifiers of physical systems near a critical point and for classifying the phase transition into universality classes.
\newpage
\section*{Conventions And Notations}
We defined these conventions and notations that are going to be used throughout this work.
\begin{itemize}
\item Natural units are used as $c = \hbar = k_b = 1$.
\item Fourier transformation $f(\omega)$ of a function $f(x)$ in a $d$- dimensional Space is is defined as
\begin{equation}\label{fourier_transform}
f(\omega) = \int\limits_{\mathbb{R}^d} dx^d e^{i\omega x}f(x) \text{   and the inverse   }f(x) = \int\limits_{\mathbb{R}^d} \frac{d\omega^d}{(2\pi)^d} e^{i\omega x}f(\omega)
\end{equation}
\item The Fourier transformation of a field described in imaginary time $\tau$ to its representations in Matsubara modes is in thermic fieldtheory defined as:
\begin{equation}\label{matsubara_transformation}
\phi_m(\vec r) = \int\limits_0^\beta d\tau e^{i\omega_m \tau}\phi(\tau,\vec r)
\end{equation}
and the inverse:
\begin{equation}\label{inverse_matsubara_transformation}
\phi(\tau,\vec r) = T\sum\limits_{m \in \mathbb{Z}}e^{-i\omega_m \tau}\phi_m (\vec r)
\end{equation}
where $\omega_m = 2\pi m T$ describes the Matsubara frequencies and $\beta = 1/T$ is the inverse Temperature.
\item In combination of the Momentum integral with the Matsubarasum we use the following short hand notation:
\begin{equation}\label{shorthand_matsubara}
\int \frac{dp^3}{(2\pi)^3}T\sum\limits_{n\in \mathbb{Z}}\equiv \sumint_p\xrightarrow{T\rightarrow 0}\int \frac{dp^4}{(2\pi)^4}
\end{equation}
\item For functional derivatives the following abbreviation is used
\begin{equation}\label{functional_derivative}
\Gamma^{(n)}[\phi](x_1,...,x_n) = \frac{\delta^{n}\Gamma}{\delta \phi(x_1)...\delta\phi(x_n)}
\end{equation}
\end{itemize}
