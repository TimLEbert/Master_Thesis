%generates the title
\begin{titlepage}
    %\drop=0.1\textheight
    \centering
    \vspace*{\baselineskip}
    \Large \\
    \large Department of Physics and Astronomy\\
    University of Heidelberg\\
    \vskip 1em%
    \rule{\textwidth}{1.6pt}\vspace*{-\baselineskip}\vspace*{2pt}
    \rule{\textwidth}{0.4pt}\\[\baselineskip]
    {\huge Differential Basis Integrale für verbessertes Orbital freies }\\[0.4\baselineskip]
    {\large \textit{English title}}\\
    {\LARGE Utilizing Differential Basis Integrals for Improved Orbital Free Machine Learned DFT and Adaptive basis function for KSDFT}\\[0.2\baselineskip]
    \rule{\textwidth}{0.4pt}\vspace*{-\baselineskip}\vspace{3.2pt}
    \rule{\textwidth}{1.6pt}\\[\baselineskip]
    \scshape
    \LARGE \textbf{Master thesis}\\\vspace{2.pt}
    {\Large in Physics}\\\vfill
    {\large submitted by}\\
    \textbf{Tim Ebert}\\\vfill
    {\large Date of submission}\\\vspace{2.pt}
    \textbf{1.11.2025}\\\vfill
    \large
    \vspace{3.2pt}
    \begin{tabular}{cc}
    Betreuer:& Prof. Dr. Fred Hamprecht\\\vspace*{4pt}
    Zweitgutachter:& Prof. Dr. Andreas Dreuw
    \end{tabular}
\end{titlepage}
\newpage
\thispagestyle{plain}
	\large
	\begin{center}
    \textbf{Zusammenfassung}
\end{center}
\normalsize
Es wurden differenzierbare Basis integrale implementiert die genutzt wurden um orbital free basis sets zu fitten und adaptive minimale Basis Sets zu implementieren. Die fähigkeiten dieser geffittenten basis functionen wurde mit anderen klassischen Basis sets verglichen 
	\vspace{5cm}
\begin{center}
	\large
    \textbf{Abstract}
\end{center}
\normalsize
The LPA truncation of the euclidean functional renormalization group (FRG) was formulated for finite temperatures for a O(N)- model with a heat bath attached. The flow equations for the effective potential and the dissipation constant $\gamma$ were determined. Using numerical simulations, the flow equation was solved. Using the data collected in the simulations the effects of dissipation on the phase transition and the flow of the dissipation constant itself were studied for varying initial parameters.


