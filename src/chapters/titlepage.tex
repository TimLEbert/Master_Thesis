%generates the title
\begin{titlepage}
    %\drop=0.1\textheight
    \centering
    \vspace*{\baselineskip}
    \Large \\
    \large Department of Physics and Astronomy\\
    University of Heidelberg\\
    \vskip 1em%
    \rule{\textwidth}{1.6pt}\vspace*{-\baselineskip}\vspace*{2pt}
    \rule{\textwidth}{0.4pt}\\[\baselineskip]
    {\huge DIfferenzierbare Basis Integrale für verbessertes Orbital freies }\\[0.4\baselineskip]
    {\large \textit{English title}}\\
    {\LARGE Utilizing Differentiable Basis Integrals for Improved Orbital Free Machine Learned DFT and Adaptive basis function for KSDFT}\\[0.2\baselineskip]
    \rule{\textwidth}{0.4pt}\vspace*{-\baselineskip}\vspace{3.2pt}
    \rule{\textwidth}{1.6pt}\\[\baselineskip]
    \scshape
    \LARGE \textbf{Master thesis}\\\vspace{2.pt}
    {\Large in Physics}\\\vfill
    {\large submitted by}\\
    \textbf{Tim Ebert}\\
    \large born in Marburg\vfill
    {\large Date of submission}\\\vspace{2.pt}
    \textbf{1.11.2025}\\\vfill
    \large
    \vspace{3.2pt}
    \begin{tabular}{cc}
    Supervisor:& Prof. Dr. Fred Hamprecht\\\vspace*{4pt}
    Second Reviewer:& Prof. Dr. Andreas Dreuw
    \end{tabular}
\end{titlepage}
\newpage
\thispagestyle{plain}
	\large
	\begin{center}
    \textbf{Abstract}
\end{center}
\normalsize
We developed and evaluated multiple density fitting methods for their use case in orbital-free density functional theory (OF-DFT). We implemented differentiable basis set integrals in PyTorch. Using these, we optimized orbital-free basis sets to generate improved training labels for machine learning-based OF-DFT . We assessed their effect on the training and accuracy of our ML-model and found that they greatly improve the predicted densities and energies without sacrificing  speed. Additionally, we leveraged the differentiable integrals to implement adaptive basis sets for Kohn-Sham DFT, using graph neural networks for per-atom basis function prediction, and compared their performance against conventional static basis sets.
	\vspace{5cm}
\begin{center}
	\large
    \textbf{Abstract}
\end{center}
\normalsize
Wir haben mehrere Dichteanpassungs methoden für ihre Anwendung in der orbitalfreien Dichtefunktionaltheorie (OF-DFT) entwickelt und bewertet. Wir implementierten differenzierbare Basissatzintegrale in PyTorch. Mit diesen optimierten wir orbitalfreie Basissätze, um verbesserte Trainingsetiketten für OF-DFT mit maschinellem Lernen zu erzeugen. Wir haben ihre Auswirkungen auf das Training und die Genauigkeit unseres ML-Modells bewertet und festgestellt, dass sie die vorhergesagten Dichten und Energien ohne Geschwindigkeitseinbußen erheblich verbessern. Darüber hinaus haben wir die differenzierbaren Integrale genutzt, um adaptive Basissätze für die Kohn-Sham-DFT zu implementieren, indem wir neuronale Graphen-Netzwerke für die Vorhersage von Basisfunktionen pro Atom verwendet haben, und ihre Leistung mit herkömmlichen statischen Basissätzen verglichen.


