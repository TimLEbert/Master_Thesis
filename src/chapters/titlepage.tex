%generates the title
\begin{titlepage}
    %\drop=0.1\textheight
    \centering
    \vspace*{\baselineskip}
    \Large Justus-Liebig-Universität Giessen\\
    \large Fachbereich 07\\
    Institut für Theoretische Physik\\
    \vskip 1em%
    \rule{\textwidth}{1.6pt}\vspace*{-\baselineskip}\vspace*{2pt}
    \rule{\textwidth}{0.4pt}\\[\baselineskip]
    {\huge Funktionale Renormierungsgruppe für selbstwechselwirkende Skalarfelder mit dissipativer Dynamik}\\[0.4\baselineskip]
    {\large \textit{English title}}\\
    {\LARGE Functional Renormation Group for Self-Interacting Scalarfields with Dissipative Dynamic}\\[0.2\baselineskip]
    \rule{\textwidth}{0.4pt}\vspace*{-\baselineskip}\vspace{3.2pt}
    \rule{\textwidth}{1.6pt}\\[\baselineskip]
    \scshape
    \LARGE \textbf{Bachelorthesis}\\\vspace{2.pt}
    {\Large im Fach Physik von}\\\vfill
    \textbf{Tim Ebert}\\\vfill
    {\large eingereicht am}\\\vspace{2.pt}
    \textbf{00000000}\\\vfill
    \large
    \vspace{3.2pt}
    \begin{tabular}{cc}
    Betreuer:& Prof. Dr. Lorenz von Smekal\\\vspace*{4pt}
    Zweitgutachter:& Prof. Dr. Christian  Fischer
    \end{tabular}
\end{titlepage}
\newpage
\thispagestyle{plain}
	\large
	\begin{center}
    \textbf{Zusammenfassung}
\end{center}
\normalsize
Es wurden differenzierbare Basis integrale implementiert die genutzt wurden um orbital free basis sets zu fitten und adaptive minimale Basis Sets zu implementieren. Die fähigkeiten dieser geffittenten basis functionen wurde mit anderen klassischen Basis sets verglichen 
	\vspace{5cm}
\begin{center}
	\large
    \textbf{Abstract}
\end{center}
\normalsize
The LPA truncation of the euclidean functional renormalization group (FRG) was formulated for finite temperatures for a O(N)- model with a heat bath attached. The flow equations for the effective potential and the dissipation constant $\gamma$ were determined. Using numerical simulations, the flow equation was solved. Using the data collected in the simulations the effects of dissipation on the phase transition and the flow of the dissipation constant itself were studied for varying initial parameters.


