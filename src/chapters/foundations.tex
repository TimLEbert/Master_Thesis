This section introduces the fundamentals that are neccesary to understand the following chapters. The section is divided into four parts. First the Density functional theory is introduced, followed by the orbital free DFT formulation. The third part introduces the basis sets that are used in DFT calculations. The last part introduces machine learning and the graphformer model that is used in this thesis.




Methods:
Density functional theory and speciffically kohn sham
Basis sets GTOs and PAOs and their integrals. typical basis sets
Orbital free dft formulation basis sets even tempered, density fitting 
Machine learing
Graphformer  Message passing Graph neural network

\section{Density functional theory}
\subsection{Kohn Sham Density Functional Theory(KS-DFT)}
The exact formulation of the Hamiltonial of the electrons of a static many body system, like a molecule, taken into account the born oppenheimer approximation of static nucleii , is given by the Schrödinger equation(using natural units):
\begin{align}
    \hat{H}\psi(\mathbf{r_1},\mathbf{r_2},\cdots,\mathbf{r_N}) =(\hat{T} + \hat V_{ee} + \hat V_{ext})\psi(\mathbf{r_1},\mathbf{r_2},\cdots,\mathbf{r_N}) = E\psi(\mathbf{r_1},\mathbf{r_2},\cdots,\mathbf{r_N})
\end{align}
With the ground state Energy and the electron density given by
\begin{align}
    E^{\star}=\min _{\psi: \text { antisym },\langle\psi \mid \psi\rangle=1}\langle\psi| \hat{T}+\hat{V}_{\mathrm{ee}}+\hat{V}_{\mathrm{ext}}|\psi\rangle\\
    \rho_{[\psi]}(\mathbf{r}):=N \int\left|\psi\left(\mathbf{r}, \mathbf{r}^{(2)}, \cdots, \mathbf{r}^{(N)}\right)\right|^2 \mathrm{~d} \mathbf{r}^{(2)} \cdots \mathrm{d} \mathbf{r}^{(N)}
\end{align}
Which can be rewritten to
\begin{align}\label{eq:energy_density}
     E^{\star} & =\min _{\rho: \geqslant 0, \int \rho(\mathbf{r}) \mathrm{d} \mathbf{r}=N}\left(\min _{\psi: \text { antisym }, \rho_{[\psi]}=\rho}\langle\psi| \hat{T}+\hat{V}_{\mathrm{ee}}|\psi\rangle\right)+E_{\text {ext }}[\rho] \\ & =\min _{\rho: \geqslant 0, \int \rho(\mathbf{r}) \mathrm{d} \mathbf{r}=N}\left\{E[\rho]:=U[\rho]+E_{\text {ext }}[\rho]\right\} .
\end{align}
However, $\psi(\mathbf{r_1},\mathbf{r_2},\cdots,\mathbf{r_N})$ is a function of $3N$ variables, rendering the equation unsolvable for systems with more than two electrons. Kohn-Sham DFT (KS-DFT) \cite{kohn_self-consistent_1965}, which extends the foundational work of Hohenberg and Kohn \cite{hohenberg_inhomogeneous_1964} on Density Functional Theory (DFT), approximates the correlated many-body wave with a determinantal wavefunction $\psi(\mathbf{r_1},\mathbf{r_2},\cdots,\mathbf{r_N}) = \text{det}(\{\phi_i(r_j)\}_{i,j=1,...,N})$ built of $N$ single-particle functions $\phi_i(\mathbf{r})$, termed orbitals. This leads to the Kohn-Sham equations:
\begin{align}
    E_{KS}[\rho] = T_S[\rho] + E_{eff}[\rho]
\end{align}
Where
\begin{align}\label{eq:kin}
    T_S[\rho] &:=\min _{\phi_i: \text {orthonormal}, \rho_{[\phi]}=\rho }\sum\limits_{i=1}^n\nabla\phi_i \nabla \phi_i\\
    E_{eff}[\rho] &:= V_{ext}[\rho] + V_H[\rho] + V_{xc}[\rho]\\
    E_H[\rho] &:= \int\int \frac{\rho(\mathbf{r})\rho(\mathbf{r'})}{|\mathbf{r}-\mathbf{r'}|}d\mathbf{r}d\mathbf{r'}\\
    E_{xc}[\rho] &:= \int \rho(\mathbf{r})\epsilon_{xc}(\rho(\mathbf{r}))d\mathbf{r}\\
    E_{ext}[\rho] &:= \int v_{ext}(\mathbf{r})\rho(\mathbf{r})d\mathbf{r}.
\end{align}
The exchange-correlation energy $E_{xc}[\rho]$ is typically approximated using empirically fitted functionals. The calculation of the effective potential in terms of the orbital free density we first need to fit the kohn sham density to the orbitals.\\
The Kohn Sham equations are solved iteratively using the Self-Consistent Field (SCF) method:
The total energy is expressed in terms of the Kohn-Sham orbitals as:
\begin{align}
    E_{KS} = \min _{\phi_i: \text {orthonormal}} \left( \sum_{i=1}^N \phi_i \Delta \phi_i + E_\text{H}[\rhoPhi]+ E_\text{XC}[\rhoPhi]+E_\text{ext}[\rhoPhi]\right)
\end{align}
To solve the kohn sham equations the Energy is varied with respect to the orbitals and the resulting equations
\begin{align}
    \frac{\delta E_{KS}}{\delta \phi_i(\mathbf{r})} = -\Delta \phi_i(\mathbf{r}) + \frac{\delta E_\text{H}[\rhoPhi]}{\delta \phi_i}(\mathbf{r}) + \frac{\delta E_\text{XC}[\rhoPhi]}{\delta \phi_i}(\mathbf{r}) + \frac{\delta E_\text{ext}[\rhoPhi]}{\delta \phi_i}(\mathbf{r}) = 0
\end{align}
 are solved iteratively using the SCF method.
\subsection{Self Consistent Field (SCF)Method}
In the SCF method the following steps are repeated until convergence is reached:
We will from now on begin to enumerate the orbitals corresponding to their interation in the scf procedure $\mathbf{\Phi}^{(j)} = \{\phi_i^{(j)}\}_{i=1}^n$, where $(j)$ denotes the current iteration.
\begin{enumerate}
    \item Start with an initial guess for $\rho(\mathbf{r})$
    \item Calculate $V_\text{eff}_i[\mathbf{\Phi}^{(j)}] = \frac{\delta E_\text{eff}[\mathbf{\Phi}^{(j)}]}{\delta \phi_i^{(j)}}(\mathbf{r})$
    \item Solve the Kohn-Sham equations while fixing $V_\text{eff}_i[\mathbf{\Phi}^{(j)}]$ to obtain the next set of orbitals $\phi_i^{(j+1)}(\mathbf{r})$.
    \item Calculate the total energy $E_{KS}$
    \item If $|E_{KS}[\phi_i^{(j+1)}] - E_{KS}[\phi_i^{(j)}]| < \epsilon_\text{tolerance}$ and $\Phi^{(j+1)}(\mathbf{r})-\Phi^{(j)}(\mathbf{r})$ is small, stop. Otherwise, return to step 2 and increase the number of the current iteration $(j)$.
\end{enumerate}
To apply this procedure to a real system, we must also consider how to represent the orbitals. For this we are using the gaussian basis sets introduced in the previous section.




















\section{Orbital free DFT}
Rooted in the original formulation of Density Functional Theory by Hohenberg and Kohn \cite{HohenbergKohn1964}, Orbital-Free Density Functional Theory (OF-DFT) aims to calculate the ground state properties of a many-electron system directly from the electron density, without the need for individual electronic orbitals.
\subsection{Theoretical Foundation}

Based on equation \eqref{eq:energy_density} the total energy of a system can be expressed as a functional of the electron density $\rho(\mathbf{r})$. The challenge of Orbtial free DFT lies in finding an accurate representation of the kinetic energy functional from equation \eqref{eq:kin} directly from the density, without resorting to orbitals. Historic attemps\cite{thakkar1992comparison,wang1999orbital} at this used numerical approximations. Known approximation include the Thomas-Fermi functional\cite{thomas_fermi_1927} and the von Weizsäcker functional\cite{von_weizsacker_1935}. These functionals were based on the kinetic energy of a non-interacting electron gas and its gradient.
\subsubsection{Thomas-Fermi Functional}
The simplest approximation is the Thomas-Fermi functional:
\begin{equation}
T_{TF}[n] = C_{TF} \int n^{5/3}(\mathbf{r}) d\mathbf{r}
\end{equation}
where $C_{TF} = \frac{3}{10}(3\pi^2)^{2/3}$.
\subsubsection{von Weizsäcker Functional}
The von Weizsäcker functional then adds a additional gradient correction:
\begin{equation}
T_{vW}[n] = \frac{1}{8} \int \frac{|\nabla n(\mathbf{r})|^2}{n(\mathbf{r})} d\mathbf{r}
\end{equation}
\subsubsection{Generalized Gradient Approximation (GGA)}
More sophisticated functionals incorporate higher-order density gradients:
\begin{equation}
T_{GGA}[n] = \int t_{GGA}(n(\mathbf{r}), \nabla n(\mathbf{r}), \nabla^2 n(\mathbf{r}), ...) d\mathbf{r}
\end{equation}

If one can represent the total energy of the system as a function of density, the ground state can be found by minimizing the total energy functional using conventional techniques like gradient descent. With the rise of machine learning in DFT, a promising approach is to learn an empirical approximation of the kinetic energy functional from a large dataset of electronic structure calculations.
While initial attempts by Remme et al. and Imoto et al. \cite{remme_kineticnet_2023,imoto2021order} used grid-based density representations, \textsc{M-OFDFT} \cite{zhang_m-ofdft_2023} pioneered the use of atom-centered basis functions, achieving the first successful prediction of energies with chemical accuracy across a large set of geometries. \textsc{M-OFDFT} also introduced the use of individual Self-Consistent Field (SCF) algorithm steps to generate model labels.
However, these labels were insufficient for the model to emulate the true functional around the ground state accurately enough to allow convergence of the minimization procedure. Instead, the density coefficients tended to drift in unphysical directions, necessitating a complicated halting criterion to generate densities near the ground state.
\subsection{OF-DFT label generation}\label{ofdft_labelgen}
In the following, we will describe the label generation process proposed by the authors of M-OFDFT \cite{zhang_m-ofdft_2023} and the modifications we made to increase training data diversity and improve model convergence.\\
In the following we will use the following conventions:\\
We denote the orbital basis functions as $\{\eta_\mu(\mathbf{r})\}_{\mu=1,...,N_\text{basis}}$, which when contracted with the coefficients $\{C_{i,\mu}\}_{i=1,..,N_\text{electrons},\mu=1,...,N_\text{basis}}$ describe the orbitals $\phi_i(\mathbf{r}) = C_i^\mu \eta_\mu(\mathbf{r})$. The orbitals can be used to calculate the electron density $\rho(\mathbf{r}) = \sum_{i=1}^{N_\text{electrons}}\phi_i(\mathbf{r})^2=\sum_{i=1}^{N_\text{electrons}}C_i^\mu \eta_\mu(\mathbf{r})C_i^\nu \eta_\nu(\mathbf{r}) =  \eta_\mu(\mathbf{r})\Gamma^{\mu,\nu} \eta_\nu(\mathbf{r})$. Where we defined the density matrix $\Gamma^{\mu,\nu} = \sum_{i=1}^{N_\text{electrons}}C_i^\mu C_i^\nu$.\\
To describe the density in the orbital free basis we use another basis set $\{\omega_\mu(\mathbf{r})\}_{\mu=1,...,N_\text{of-basis}}$ with corresponding coefficients $\mathbf{p} =\{p^\mu\}_{\mu=1,...,N_\text{of-basis}}$ which form a linear basis for the density $\rho_\text{OF}(\mathbf{r}) = p^\mu \omega_\mu(\mathbf{r})$.\\
We also use the follwing symbols to denote common basis integrals between the two basis sets following the conventions introduced in section \ref{integral_notation}:
\begin{align}
    W_{\mu\nu} &= \langle\omega_\mu |\omega_\nu\rangle\\
    L_{\mu,\nu\gamma} &= \langle \omega_\mu |\eta_\nu\eta_\gamma\rangle\\
    D_{\alpha,\beta,\gamma,\delta}  &= \langle\eta_\alpha\eta_\beta|\eta_\gamma\eta_\delta\rangle\\
    \tilde{W}_{\mu\nu} &= (\omega_\mu |\omega_\nu)\\
    \tilde{L}_{\mu,\nu\gamma} &= (\omega_\mu |\eta_\nu\eta_\gamma)\\
    \tilde{D}_{\alpha,\beta,\gamma,\delta}  &= (\eta_\alpha\eta_\beta|\eta_\gamma\eta_\delta)\\
    S_{\alpha,\beta} &= \langle \eta_\alpha|\eta_\beta\rangle\\
    {v_{ext}}_\mu &= \int \omega_\mu (r) V_{ext} (r)dr\\
    {V_{ext}}_{\mu,\nu} &= \langle \eta_\mu | V_{ext} (r) | \eta_\nu \rangle\\
\end{align}
We are now set to describe the label generation process.\\
\subsubsection{Density fitting}\label{into_density_fitting}
Density fitting is a procedure to represent the density produced by the Kohn-Sham orbitals using orbital-free basis functions $\{\omega_\mu(\mathbf{r})\}_\textit{\mu=1,...,N_\text{of-basis}}$, while preserving its electronic properties. This mean calculating $\mathbf{p}(\{C_{i,\mu}\}_{i=1,..,N_\text{electrons},\mu=1,...,N_\text{basis}})$.\\
In chapter \ref{chapter:densityfitting}, we define and evaluate different density fitting algorithms and compare the properties of the densities they produce.\\
\subsubsection{Energy and gradient label generation}
The label for the kinetic energy in coordinates is indirectly calculated by:

\begin{align}
    T_S(\mathbf{p}) = T_S(C) + E_{H}(C) + E_{XC}(C) + E_{ext}(C) - E_{H}(\mathbf{p}) + E_{XC}(\mathbf{p}) + E_{ext}(\mathbf{p}).
\end{align}

We cannot calculate the gradient of the kinetic energy directly from this equation as the orbital free density coefficients $\mathbf{p}$ are dependent on
the coefficients in the orbital basis $\{C_{i,\mu}\}_{i=1,..,N_\text{electrons},\mu=1,...,N_\text{basis}}$ in a nontrivial way.\\
Instead, we make use of the minimization procedure in the KSDFT calculation. Additionally we use a method developed by Manuel Klockow which in each iteration of the self consistent field procedere adds an additional pertubation to the effective potential to increase the diversity of the densities in the labels. In each iteration the next set of orbitals is then computed as follows:

\begin{align}
    \Phi^{k} = \underset{\{\phi_i\}_{i=1}^n \text{orthonormal}}{\text{argmin}} \langle \psi_{\mathbf{\phi}} | \hat T_S | \psi_{\mathbf{\phi}} \rangle + \sum_{k'<k} \pi^{(k')}_k V_{eff}^{k'}[\rho_{[\mathbf{\phi}]}] + V_\text{peturb}^{k}(\rhoPhi)
\end{align}


Where $\pi^{(i)}_k$ are the DIIS coefficients of the KSDFT calculation and

\begin{align}
    V_{eff}^{k'}[\rho_{[\mathbf{\phi}]}] = \int \rho_{[\mathbf{\phi}]}(\mathbf{r}) V_{eff[\rho_{[\mathbf{\phi}^{k'}]}]}(r)dr\\
    V_\text{peturb}^{k}(\rhoPhi) = \int \rhoPhi(\mathbf{r}) v_{\text{peturb},\mu}^{k}\omega_\mu(\mathbf{r})dr
\end{align}
is the effective potential of the $k'$-th iteration integrated over the density of the new density and the pertubation of the effective potential of the $k$th iteration. With $v_{\text{peturb},\mu}^{k} \sim \epsilon_k * \mathcal{N}(1,0)$ and $\omega_\mu$ being basis function in the orbital free basis.
$\epsilon_k$ is then chosen such that starts at the 5 interation and decreases over time.
This is solved using Laplace multipliers:

\begin{align}
    \frac{\delta T_S[\rho_{[\mathbf{\phi}^{k}]}]}{\delta \rho}(\mathbf{r}) + \sum_{k'<k} \pi^{(k')}_k V_{eff}^{k'}(\mathbf{r}) + v_{\text{peturb},\mu}^{k}\omega_\mu(\mathbf{r})= \mu^{(k)}
\end{align}
From here we cannot compute the gradient of the kinetic energy exactly but we can calculate its projection in the plane of constant particle numbers. As in density optimisation our searchspace is restricted to this plane this is sufficient for our purposes.
The projected gradient of the kinetic energy is then calculated as the DIIS weighted average over the projected last few effective potentials:

\begin{align}
    \nabla_\mathbf{p} T_S(\mathbf{p}) &= \int \frac{\delta T_S[\rho_{[\mathbf{\phi}^{k}]}]}{\delta \rho}(\mathbf{r}) \mathbf{\omega}(\mathbf{r}) d\mathbf{r} \\
    &= -\int\sum_{k'<k} \pi^{(k')}_k V_{eff}^{k'}(\mathbf{r}) \mathbf{\omega}v_{peturb,\mu}^{k}\omega_\mu(\mathbf{r})(\mathbf{r}) +\mu^{(k)}\mathbf{\omega}(\mathbf{r})d\mathbf{r} = -{\mathbf{v}}_{eff\{\mathbf{p}^{k'}\}_{k'< k}}+\mu^{(k)}\mathbf{w}\\
    \left( \textbf{I}-\frac{\textbf{w}^{(d)}{\textbf{w}^{(d)}}^T}{{\textbf{w}^{(d)}}^T
        \textbf{w}^{(d)}}\right) \nabla_\mathbf{p} T_S(\mathbf{p}) &= -\left( \textbf{I}-\frac{\textbf{w}^{(d)}{\textbf{w}^{(d)}}^T}{{\textbf{w}^{(d)}}^T
        \textbf{w}^{(d)}}\right){\mathbf{v}}_{eff\{\mathbf{p}^{k'}\}_{k'< k}}\\
    \mathbf{v}_{eff}(\mathbf{p}) &= \nabla_\mathbf{p} (E_{H}(\mathbf{p}) + E_{XC}(\mathbf{p}) + E_{ext}(\mathbf{p}))
\end{align}
These labels for the kinetic energy are then used to train a machine learning model to emulate the kinetic energy functional.\\

\subsection{Preparation of the data}
To prepare the data to be better suited for a machine learning Model we employ the same techniques as in the original M-OFDFT\cite{zhang_m-ofdft_2023} paper, therefor we only quickly notion their functions here.
\subsubsection{Natrep}
Orthogonolizing the OF-basis for more meaningfull coefficients.
\subsubsection{Dimensionwise rescaling}
An statistic is calculated for the coefficients of each atomtype in the dataset and the coefficients are rescaled to have a mean of 0 and a standard deviation of 1.
\subsubsection{Local frames}
To make the model invariant to rotations is for each atom a local coordinate system is choosen which is directed in the direction of its nearest neightbor. Then the coefficients are projected into this local frame.

\section{Basis sets}
In order to Represent orbitals inside a Kohn Sham DFT calculation atom centric basis sets are the norm.
These basis sets usually consist of a radial part times a radial distribution function.
\subsection{Polarized Atomic Orbitals}
Polarized Atomic Orbitals(PAOs) are derived as a linear combination from a larger primary basis of functions centered on individuals atom centers. Formally we can write a PAO basis function $\eta_{\mu}(\mathbf{r})$ as a weighted sum of the primary basis functions $\tilde\eta_\nu(\mathbf{r})$ where $\mu,\nu$ are belonging to the same Atom.
\begin{align}
    \tilde \eta_{\mu} &=\sum\limits_{\nu} A_{\mu,\nu} \eta_{\nu} 
\end{align}
\subsection{Gaussian basis functions}
The the radial individal basis functions $\eta_{\mu}(\mathbf{r})$ are usually chosen either as Gaussian
's centered on the same atom at position($\mathbf{R}$) with different exponents and multiplied with spherical harmonics$Y_{lm}$ to accomplish some angular dependency.
\begin{align}
    \eta_{\mu}(\mathbf{r}) &= N(\alpha,m,l) ||\mathbf{r}-\mathbf{R}||^l Y_{lm}(\frac{\mathbf{r}-\mathbf{R}}{||\mathbf{r}-\mathbf{R}||}) e^{-\alpha ||\mathbf{r}-\mathbf{R}||^2}
\end{align}
The advantages of Gaussian orbitals are that they are computationally efficient, are easily integrated and the product of two gaussians is again a gaussian which simplifies many otherwise very complex integrals.
The spherical harmonics on the other side enables arbitrary angular resolution and can be formulated in their Cartesian Formulation as an polynominal of just order $l$.
\begin{align}
    ||\mathbf{v}||^l Y_{lm}(\frac{\mathbf{v}}{||\mathbf{v}||}) &= \sum\limits_{a,b,c\in \mathbb{N}_0 a+b+c\leq l} A_{l,a,b,c} v_1^a v_2^b v_3^c
\end{align}
Where $A_{l,a,b,c}$ are constants. This leads to PAOs just being finite sums of finite polynominals multiplied with gaussians.

\subsection{Integrals}
Most relevant integrals for DFT involve integrals over products of gaussians which can be computed analytically. Examples are the overlap integral $\langle \eta_{\mu}|\eta_{\nu}\rangle$, the kinetic energy integral $\langle \nabla \eta_{\mu}|\nabla \eta_{\nu}\rangle$, the electron-nucleus integral $\langle \eta_{\mu}|\frac{Z}{||\mathbf{r}-\mathbf{R}||}|\eta_{\nu}\rangle$ as well as the hartree integral $(\eta_{\mu}\eta_{\nu}|\eta_\gamma\eta_\delta)$. Other more complicated integrals, such as the Thomas Fermi functional or the von Weizsäcker functional, as well as practically all exchange correlation functionals, are usually computed numerically on a sufficently large grid.
The way these integrals are usually calculated can be demonstrated on the example of the overlap integral. First one can notice that it is possible to write the product of the two gaussians as a single gaussian and simplify.
\begin{align}
    \langle \phi|\psi \rangle &= \langle   \sum\limits_{a,b,c\in \mathbb{N}_0 a+b+c\leq l_1} A_{l,a,b,c} (v_1-x_1)^a (v_2-x_2)^b (v_3-x_3)^c e^{-\alpha (\mathbf v -\mathbf x)^2}| \sum\limits_{a,b,c\in \mathbb{N}_0 d+e+f\leq l_2} B_{l,d,e,f} (w_1-x_1)^d (w_2-x_2)^e (w_3-x_3)^{f}e^{-\beta(\mathbf w -\mathbf x)^2}\rangle\\
    &=\int \sum\limits_{a,b,c\in \mathbb{N}_0 a+b+c\leq l_1+l_2} C_{l,a,b,c} (x_1)^a (x_2)^b (x_3)^{c}e^{-(\alpha+\beta)\alpha(\frac{\alpha \mathbf v +\beta\mathbf w}{\alpha+\beta} -\mathbf x)^2-\frac{\alpha\beta}{\alpha+\beta}(\mathbf{v}-\mathbf{w})^2}\\
    &=\sum\limits_{a,b,c\in \mathbb{N}_0 a+b+c\leq l_1+l_2} C_{l,a,b,c} I_{a}(\alpha,\beta,(\mathbf{v}-\mathbf{w})|_{1})I_{b}(\alpha,\beta,(\mathbf{v}-\mathbf{w})|_{2})I_{c}(\alpha,\beta,(\mathbf{v}-\mathbf{w})|_{3})
\end{align}
The individual integrals $I_{a}(\alpha,\beta,x)$ can be computed using recursion relations the recursion root can be camputed as:
\begin{align}
    I_{0}(\alpha,\beta,\mathbf{v}-\mathbf{w}) &= \sqrt{\frac{\pi}{\alpha+\beta}} e^{-\frac{\alpha\beta}{\alpha+\beta}(\mathbf{v}-\mathbf{w})^2}
\end{align}
And the higher order terms as:
\begin{align}
    I_{a,b,c}(\alpha,\beta,\mathbf{v}-\mathbf{w}) &= \frac{1}{2(\alpha+\beta)}\left( (a-1)I_{a-1,b,c}(\alpha,\beta,\mathbf{v}-\mathbf{w}) + (b-1)I_{a,b-1,c}(\alpha,\beta,\mathbf{v}-\mathbf{w})\right.\\
    &\left.+ (c-1)I_{a,b,c-1}(\alpha,\beta,\mathbf{v}-\mathbf{w})\right)
\end{align}
There exist more recursion relations for the other types of integrals.
\section{Machine Learning}
Machine Learning is since .. Nobel price 2024


\section{Graph Neural Networks}
Graph neural networks are a special Variant of Neural networks that apply to graph shaped data.
In our case the graph is the molecular structure of the system that we want to model. Each atom functions as a node in the graph which contains information about the local density around this atom in form of the coefficients of the basis function that centered on the atom. The Edges of the graph contain information in form of the distance towards the next atom. These distances can be embedded in the edge features of the graph neural network.