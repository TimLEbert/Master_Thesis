This section introduces the fundamentals that are neccesary to understand the following chapters. The section is divided into four parts. First the Density functional theory is introduced, followed by the orbital free DFT formulation. The third part introduces the basis sets that are used in DFT calculations. The last part introduces machine learning and the graphformer model that is used in this thesis.




Methods:
Density functional theory and speciffically kohn sham
Basis sets GTOs and PAOs and their integrals. typical basis sets
Orbital free dft formulation basis sets even tempered, density fitting 
Machine learing
Graphformer  Message passing Graph neural network


\section{Density functional theory}
\subsection{Historical Background}
Kohn-Sham Density Functional Theory (KSDFT) stands as a cornerstone in computational quantum mechanics, revolutionizing our approach to electronic structure calculations. Developed by Walter Kohn and Lu Jeu Sham in 1965 \cite{KohnSham1965}, KSDFT extends the foundational work of Hohenberg and Kohn \cite{HohenbergKohn1964} on Density Functional Theory (DFT).
\subsection{Theoretical Foundation}
The theoretical underpinning of DFT rests on two fundamental theorems proved by Hohenberg and Kohn:
\begin{theorem}
The ground-state properties of a many-electron system are uniquely determined by the electron density $n(\mathbf{r})$.
\end{theorem}
\begin{theorem}
There exists a universal functional of the electron density, $F[n(\mathbf{r})]$, which can be used to find the ground-state energy of the system.
\end{theorem}
These theorems establish that the ground-state energy of a system can be expressed as a functional of the electron density:
\begin{equation}
E[n] = F[n] + \int V_{\text{ext}}(\mathbf{r})n(\mathbf{r})d\mathbf{r}
\end{equation}
where $V_{\text{ext}}(\mathbf{r})$ is the external potential and $F[n]$ is a universal functional independent of the external potential.
\subsection{The Kohn-Sham Approach}
Kohn and Sham proposed a practical approach to apply DFT by introducing a fictitious system of non-interacting particles that generate the same density as the system of interacting particles. This approach involves solving a set of single-particle Schrödinger-like equations, known as the Kohn-Sham equations:
\begin{equation}
\left[-\frac{1}{2}\nabla^2 + V_{\text{eff}}(\mathbf{r})\right]\phi_i(\mathbf{r}) = \epsilon_i\phi_i(\mathbf{r})
\end{equation}
where $\phi_i(\mathbf{r})$ are the Kohn-Sham orbitals and $\epsilon_i$ are their corresponding energies. The effective potential $V_{\text{eff}}(\mathbf{r})$ is defined as:
\begin{equation}
V_{\text{eff}}(\mathbf{r}) = V_{\text{ext}}(\mathbf{r}) + V_{\text{H}}(\mathbf{r}) + V_{\text{xc}}(\mathbf{r})
\end{equation}
Here, $V_{\text{ext}}(\mathbf{r})$ is the external potential, $V_{\text{H}}(\mathbf{r})$ is the Hartree potential, and $V_{\text{xc}}(\mathbf{r})$ is the exchange-correlation potential.
\subsection{Contributions to the Kohn-Sham Energy}
The total energy in the Kohn-Sham formulation can be expressed as:
\begin{equation}
E_{\text{KS}} = T_s[n] + E_{\text{H}}[n] + E_{\text{xc}}[n] + E_{\text{ext}}[n]
\end{equation}
Let us examine each term in detail:
\subsubsection{Kinetic Energy of Non-interacting Electrons}
The kinetic energy of the non-interacting electrons, $T_s[n]$, is given by:
\begin{equation}
T_s[n] = -\frac{1}{2}\sum_i \int \phi_i^*(\mathbf{r})\nabla^2\phi_i(\mathbf{r})d\mathbf{r}
\end{equation}
This term represents the kinetic energy of the Kohn-Sham orbitals.
\subsubsection{Hartree Energy}
The Hartree energy, $E_{\text{H}}[n]$, represents the classical electrostatic interaction energy of the electron density:
\begin{equation}
E_{\text{H}}[n] = \frac{1}{2}\int\int \frac{n(\mathbf{r})n(\mathbf{r'})}{|\mathbf{r}-\mathbf{r'}|}d\mathbf{r}d\mathbf{r'}
\end{equation}
\subsubsection{Exchange-Correlation Energy}
The exchange-correlation energy, $E_{\text{xc}}[n]$, encapsulates all many-body effects beyond the Hartree approximation. Its exact form is unknown, and developing accurate approximations for this term is a central challenge in DFT. Common approximations include:
\begin{itemize}
\item Local Density Approximation (LDA):
\begin{equation}
E_{\text{xc}}^{\text{LDA}}[n] = \int n(\mathbf{r})\epsilon_{\text{xc}}(n(\mathbf{r}))d\mathbf{r}
\end{equation}
where $\epsilon_{\text{xc}}(n)$ is the exchange-correlation energy per particle of a uniform electron gas of density $n$.
\item Generalized Gradient Approximation (GGA):
\begin{equation}
E_{\text{xc}}^{\text{GGA}}[n] = \int f(n(\mathbf{r}), |\nabla n(\mathbf{r})|)d\mathbf{r}
\end{equation}
where $f$ is a function of both the density and its gradient.
\end{itemize}
\subsection{External Potential Energy}
The external potential energy, $E_{\text{ext}}[n]$, represents the interaction of the electrons with the external potential (typically due to the nuclei):
\begin{equation}
E_{\text{ext}}[n] = \int V_{\text{ext}}(\mathbf{r})n(\mathbf{r})d\mathbf{r}
\end{equation}
\section{Self-Consistent Field Method}
The Kohn-Sham equations are solved iteratively using the Self-Consistent Field (SCF) method:
\begin{enumerate}
\item Start with an initial guess for $n(\mathbf{r})$
\item Calculate $V_{\text{eff}}(\mathbf{r})$
\item Solve the Kohn-Sham equations to obtain $\phi_i(\mathbf{r})$
\item Calculate a new density: $n_{\text{new}}(\mathbf{r}) = \sum_i |\phi_i(\mathbf{r})|^2$
\item If $|n_{\text{new}}(\mathbf{r}) - n(\mathbf{r})| < \text{tolerance}$, stop. Otherwise, return to step 2 with $n(\mathbf{r}) = n_{\text{new}}(\mathbf{r})$
\end{enumerate}

\section{Orbital free DFT}
Rooted in the original formulation of Density Functional Theory by Hohenberg and Kohn \cite{HohenbergKohn1964}, Orbital-Free Density Functional Theory (OF-DFT) aims to calculate the ground state properties of a many-electron system directly from the electron density, without the need for individual electronic orbitals.
\subsection{Theoretical Foundation}

Based on equation \eqref{eq:energy_density} the total energy of a system can be expressed as a functional of the electron density $\rho(\mathbf{r})$. The challenge of Orbtial free DFT lies in finding an accurate representation of the kinetic energy functional from equation \eqref{eq:kin} directly from the density, without resorting to orbitals. Historic attemps\cite{thakkar1992comparison,wang1999orbital} at this used numerical approximations. Known approximation include the Thomas-Fermi functional\cite{thomas_fermi_1927} and the von Weizsäcker functional\cite{von_weizsacker_1935}. These functionals were based on the kinetic energy of a non-interacting electron gas and its gradient.
\subsubsection{Thomas-Fermi Functional}
The simplest approximation is the Thomas-Fermi functional:
\begin{equation}
T_{TF}[n] = C_{TF} \int n^{5/3}(\mathbf{r}) d\mathbf{r}
\end{equation}
where $C_{TF} = \frac{3}{10}(3\pi^2)^{2/3}$.
\subsubsection{von Weizsäcker Functional}
The von Weizsäcker functional then adds a additional gradient correction:
\begin{equation}
T_{vW}[n] = \frac{1}{8} \int \frac{|\nabla n(\mathbf{r})|^2}{n(\mathbf{r})} d\mathbf{r}
\end{equation}
\subsubsection{Generalized Gradient Approximation (GGA)}
More sophisticated functionals incorporate higher-order density gradients:
\begin{equation}
T_{GGA}[n] = \int t_{GGA}(n(\mathbf{r}), \nabla n(\mathbf{r}), \nabla^2 n(\mathbf{r}), ...) d\mathbf{r}
\end{equation}

If one can represent the total energy of the system as a function of density, the ground state can be found by minimizing the total energy functional using conventional techniques like gradient descent. With the rise of machine learning in DFT, a promising approach is to learn an empirical approximation of the kinetic energy functional from a large dataset of electronic structure calculations.
While initial attempts by Remme et al. and Imoto et al. \cite{remme_kineticnet_2023,imoto2021order} used grid-based density representations, \textsc{M-OFDFT} \cite{zhang_m-ofdft_2023} pioneered the use of atom-centered basis functions, achieving the first successful prediction of energies with chemical accuracy across a large set of geometries. \textsc{M-OFDFT} also introduced the use of individual Self-Consistent Field (SCF) algorithm steps to generate model labels.
However, these labels were insufficient for the model to emulate the true functional around the ground state accurately enough to allow convergence of the minimization procedure. Instead, the density coefficients tended to drift in unphysical directions, necessitating a complicated halting criterion to generate densities near the ground state.
\subsection{OF-DFT label generation}\label{ofdft_labelgen}
In the following, we will describe the label generation process proposed by the authors of M-OFDFT \cite{zhang_m-ofdft_2023} and the modifications we made to increase training data diversity and improve model convergence.\\
In the following we will use the following conventions:\\
We denote the orbital basis functions as $\{\eta_\mu(\mathbf{r})\}_{\mu=1,...,N_\text{basis}}$, which when contracted with the coefficients $\{C_{i,\mu}\}_{i=1,..,N_\text{electrons},\mu=1,...,N_\text{basis}}$ describe the orbitals $\phi_i(\mathbf{r}) = C_i^\mu \eta_\mu(\mathbf{r})$. The orbitals can be used to calculate the electron density $\rho(\mathbf{r}) = \sum_{i=1}^{N_\text{electrons}}\phi_i(\mathbf{r})^2=\sum_{i=1}^{N_\text{electrons}}C_i^\mu \eta_\mu(\mathbf{r})C_i^\nu \eta_\nu(\mathbf{r}) =  \eta_\mu(\mathbf{r})\Gamma^{\mu,\nu} \eta_\nu(\mathbf{r})$. Where we defined the density matrix $\Gamma^{\mu,\nu} = \sum_{i=1}^{N_\text{electrons}}C_i^\mu C_i^\nu$.\\
To describe the density in the orbital free basis we use another basis set $\{\omega_\mu(\mathbf{r})\}_{\mu=1,...,N_\text{of-basis}}$ with corresponding coefficients $\mathbf{p} =\{p^\mu\}_{\mu=1,...,N_\text{of-basis}}$ which form a linear basis for the density $\rho_\text{OF}(\mathbf{r}) = p^\mu \omega_\mu(\mathbf{r})$.\\
We also use the follwing symbols to denote common basis integrals between the two basis sets following the conventions introduced in section \ref{integral_notation}:
\begin{align}
    W_{\mu\nu} &= \langle\omega_\mu |\omega_\nu\rangle\\
    L_{\mu,\nu\gamma} &= \langle \omega_\mu |\eta_\nu\eta_\gamma\rangle\\
    D_{\alpha,\beta,\gamma,\delta}  &= \langle\eta_\alpha\eta_\beta|\eta_\gamma\eta_\delta\rangle\\
    \tilde{W}_{\mu\nu} &= (\omega_\mu |\omega_\nu)\\
    \tilde{L}_{\mu,\nu\gamma} &= (\omega_\mu |\eta_\nu\eta_\gamma)\\
    \tilde{D}_{\alpha,\beta,\gamma,\delta}  &= (\eta_\alpha\eta_\beta|\eta_\gamma\eta_\delta)\\
    S_{\alpha,\beta} &= \langle \eta_\alpha|\eta_\beta\rangle\\
    {v_{ext}}_\mu &= \int \omega_\mu (r) V_{ext} (r)dr\\
    {V_{ext}}_{\mu,\nu} &= \langle \eta_\mu | V_{ext} (r) | \eta_\nu \rangle
\end{align}
We are now set to describe the label generation process.\\
\subsubsection{Density fitting}\label{into_density_fitting}
Density fitting is a procedure to represent the density produced by the Kohn-Sham orbitals using orbital-free basis functions $\{\omega_\mu(\mathbf{r})\}_\textit{\mu=1,...,N_\text{of-basis}}$, while preserving its electronic properties. This mean calculating $\mathbf{p}(\{C_{i,\mu}\}_{i=1,..,N_\text{electrons},\mu=1,...,N_\text{basis}})$.\\
In chapter \ref{chapter:densityfitting}, we define and evaluate different density fitting algorithms and compare the properties of the densities they produce.\\
\subsubsection{Energy and gradient label generation}
The label for the kinetic energy in coordinates is indirectly calculated by:

\begin{align}
    T_S(\mathbf{p}) = T_S(C) + E_{H}(C) + E_{XC}(C) + E_{ext}(C) - E_{H}(\mathbf{p}) + E_{XC}(\mathbf{p}) + E_{ext}(\mathbf{p}).
\end{align}

We cannot calculate the gradient of the kinetic energy directly from this equation as the orbital free density coefficients $\mathbf{p}$ are dependent on
the coefficients in the orbital basis $\{C_{i,\mu}\}_{i=1,..,N_\text{electrons},\mu=1,...,N_\text{basis}}$ in a nontrivial way.\\
Instead, we make use of the minimization procedure in the KSDFT calculation. Additionally we use a method developed by Manuel Klockow which in each iteration of the self consistent field procedere adds an additional pertubation to the effective potential to increase the diversity of the densities in the labels. In each iteration the next set of orbitals is then computed as follows:

\begin{align}
    \Phi^{k} = \underset{\{\phi_i\}_{i=1}^n \text{orthonormal}}{\text{argmin}} \langle \psi_{\mathbf{\phi}} | \hat T_S | \psi_{\mathbf{\phi}} \rangle + \sum_{k'<k} \pi^{(k')}_k V_{eff}^{k'}[\rho_{[\mathbf{\phi}]}] + V_\text{peturb}^{k}(\rhoPhi)
\end{align}


Where $\pi^{(i)}_k$ are the DIIS coefficients of the KSDFT calculation and

\begin{align}
    V_{eff}^{k'}[\rho_{[\mathbf{\phi}]}] = \int \rho_{[\mathbf{\phi}]}(\mathbf{r}) V_{eff[\rho_{[\mathbf{\phi}^{k'}]}]}(r)dr\\
    V_\text{peturb}^{k}(\rhoPhi) = \int \rhoPhi(\mathbf{r}) v_{\text{peturb},\mu}^{k}\omega_\mu(\mathbf{r})dr
\end{align}
is the effective potential of the $k'$-th iteration integrated over the density of the new density and the pertubation of the effective potential of the $k$th iteration. With $v_{\text{peturb},\mu}^{k} \sim \epsilon_k * \mathcal{N}(1,0)$ and $\omega_\mu$ being basis function in the orbital free basis.
$\epsilon_k$ is then chosen such that starts at the 5 interation and decreases over time.
This is solved using Laplace multipliers:

\begin{align}
    \frac{\delta T_S[\rho_{[\mathbf{\phi}^{k}]}]}{\delta \rho}(\mathbf{r}) + \sum_{k'<k} \pi^{(k')}_k V_{eff}^{k'}(\mathbf{r}) + v_{\text{peturb},\mu}^{k}\omega_\mu(\mathbf{r})= \mu^{(k)}
\end{align}
From here we cannot compute the gradient of the kinetic energy exactly but we can calculate its projection in the plane of constant particle numbers. As in density optimisation our searchspace is restricted to this plane this is sufficient for our purposes.
The projected gradient of the kinetic energy is then calculated as the DIIS weighted average over the projected last few effective potentials:

\begin{align}
    \nabla_\mathbf{p} T_S(\mathbf{p}) &= \int \frac{\delta T_S[\rho_{[\mathbf{\phi}^{k}]}]}{\delta \rho}(\mathbf{r}) \mathbf{\omega}(\mathbf{r}) d\mathbf{r} \\
    &= -\int\sum_{k'<k} \pi^{(k')}_k V_{eff}^{k'}(\mathbf{r}) \mathbf{\omega}v_{peturb,\mu}^{k}\omega_\mu(\mathbf{r})(\mathbf{r}) +\mu^{(k)}\mathbf{\omega}(\mathbf{r})d\mathbf{r} = -{\mathbf{v}}_{eff\{\mathbf{p}^{k'}\}_{k'< k}}+\mu^{(k)}\mathbf{w}\\
    \left( \textbf{I}-\frac{\textbf{w}^{(d)}{\textbf{w}^{(d)}}^T}{{\textbf{w}^{(d)}}^T
        \textbf{w}^{(d)}}\right) \nabla_\mathbf{p} T_S(\mathbf{p}) &= -\left( \textbf{I}-\frac{\textbf{w}^{(d)}{\textbf{w}^{(d)}}^T}{{\textbf{w}^{(d)}}^T
        \textbf{w}^{(d)}}\right){\mathbf{v}}_{eff\{\mathbf{p}^{k'}\}_{k'< k}}\\
    \mathbf{v}_{eff}(\mathbf{p}) &= \nabla_\mathbf{p} (E_{H}(\mathbf{p}) + E_{XC}(\mathbf{p}) + E_{ext}(\mathbf{p}))
\end{align}
These labels for the kinetic energy are then used to train a machine learning model to emulate the kinetic energy functional.\\

\subsection{Preparation of the data}
To prepare the data for machine learning, we employ the same techniques as in the original M-OFDFT \cite{zhang_m-ofdft_2023} paper. We briefly describe their functions here.
\subsubsection{Natrep}
Orthogonalizing the orbital-free basis for more meaningful coefficients.
\subsubsection{Dimensionwise rescaling}
Statistics are calculated for the coefficients of each atom type in the dataset, and the coefficients are rescaled to have a mean of 0 and a standard deviation of 1.
\subsubsection{Local frames}
To make the model rotationally invariant, a local coordinate system is defined for each atom, oriented towards its nearest neighbor. The coefficients are then projected into this local frame.

\section{Basis sets}
In order to represent spacial functions $f:\mathbb{R}^3\right \mathbb{C}$ inside a DFT calculation atom centric basis sets are most commonly used, as they provide anhanced resolution around the nuleii where most of the electron density is concentrated.
\subsection{Contracted atomic orbitals}
Contracted atomic orbitals are defined as a linear combination of a larger primary basis of basis functions centered on atom centers. Formally we can write a contracted atomic orbitals basis function $\eta_{\mu}(\mathbf{r})$ as a weighted sum of the primary basis functions $\tilde\eta_\nu(\mathbf{r})$ where $\mu,\nu$ are belonging to the same Atom.
\begin{align}
    \eta_{\mu} &=\sum\limits_{\nu} A_{\mu,\nu} \tilde \eta_{\nu}
\end{align}
\subsection{Contracted Gaussian Type Orbitals (CGTOs)}
The radial distribution of primary basis functions $\eta_{\mu}(\mathbf{r})$ is typically chosen as Gaussians centered on the same atom at position $\mathbf{R}$ with different exponents, multiplied by spherical harmonics $Y_{lm}$ to introduce angular dependency.
\begin{align}
\eta_{\mu}(\mathbf{r}) &= N(\alpha,m,l) ||\mathbf{r}-\mathbf{R}||^l Y_{lm}(\frac{\mathbf{r}-\mathbf{R}}{||\mathbf{r}-\mathbf{R}||}) e^{-\alpha ||\mathbf{r}-\mathbf{R}||^2}
\end{align}
where $N(\alpha,m,l)$ are normalisation constants.
Gaussian orbitals offer several advantages: they are computationally efficient, easily integrated, and the product of two Gaussians remains a Gaussian, which simplifies many otherwise complex integrals. However, a single Gaussian cannot accurately describe the electron density cusp near the nucleus. To address this, primitive Gaussians are typically contracted to Contracted Gaussian-Type Orbitals (CGTOs) to form a more physically representative basis set. The contraction coefficients $A_{\mu,\nu}$ are fitted to more accurate basis functions like Slater-type orbitals (STOs).
Spherical harmonics provide arbitrary angular resolution and can be formulated in their Cartesian representation as a polynomial of order $l$.
\begin{align}
    ||\mathbf{v}||^l Y_{lm}(\frac{\mathbf{v}}{||\mathbf{v}||}) &= \sum\limits_{a,b,c\in \mathbb{N}_0 a+b+c\leq l} A_{l,a,b,c} v_1^a v_2^b v_3^c
\end{align}
Where $A_{l,a,b,c}$ are constants. This leads to CGTOs just being finite sums of finite polynominals multiplied with gaussians.

\subsection{Integrals}
Most relevant integrals for DFT involve integrals over products of gaussians which can be computed analytically. Examples are the overlap integral $\langle \eta_{\mu}|\eta_{\nu}\rangle$, the kinetic energy integral $\langle \nabla \eta_{\mu}|\nabla \eta_{\nu}\rangle$, the electron-nucleus integral $\langle \eta_{\mu}|\frac{Z}{||\mathbf{r}-\mathbf{R}||}|\eta_{\nu}\rangle$ as well as the hartree integral $(\eta_{\mu}\eta_{\nu}|\eta_\gamma\eta_\delta)$. While these closed form solutions exist they still involve many repeated recursion relations and can be very compute intensive for larger molecules. Other more complicated integrals, such as the Thomas Fermi functional or the von Weizsäcker functional, as well as practically all exchange correlation functionals, are usually computed numerically on a sufficently large grid. In this work we are using the implementationa of the libcint library \cite{sun_libcint_2015} to compute these integrals, when they don't have to be differentiable.


\section{Machine Learning}